\documentclass[a4paper]{article}

\def\npart{III}
\def\nterm {Lent}
\def\nyear {2017-2018}
\def\nlecturer {Brian}
\def\ncourse {Physics}

\makeatletter
\ifx \nauthor\undefined
  \def\nauthor{Dexter Chua}
\else
\fi

\author{Based on lectures by \nlecturer \\\small Notes taken by \nauthor}
\date{\nterm\ \nyear}

\usepackage{alltt}
\usepackage{amsfonts}
\usepackage{amsmath}
\usepackage{amssymb}
\usepackage{amsthm}
\usepackage{booktabs}
\usepackage{caption}
\usepackage{enumitem}
\usepackage{fancyhdr}
\usepackage{graphicx}
\usepackage{mathdots}
\usepackage{mathtools}
\usepackage{microtype}
\usepackage{multirow}
\usepackage{pdflscape}
\usepackage{pgfplots}
\usepackage{siunitx}
\usepackage{slashed}
\usepackage{tabularx}
\usepackage{tikz}
\usepackage{tkz-euclide}
\usepackage[normalem]{ulem}
\usepackage[all]{xy}
\usepackage{imakeidx}

\makeindex[intoc, title=Index]
\indexsetup{othercode={\lhead{\emph{Index}}}}

\ifx \nextra \undefined
  \usepackage[pdftex,
    hidelinks,
    pdfauthor={Dexter Chua},
    pdfsubject={Cambridge Maths Notes: Part \npart\ - \ncourse},
    pdftitle={Part \npart\ - \ncourse},
  pdfkeywords={Cambridge Mathematics Maths Math \npart\ \nterm\ \nyear\ \ncourse}]{hyperref}
  \title{Part \npart\ --- \ncourse}
\else
  \usepackage[pdftex,
    hidelinks,
    pdfauthor={Dexter Chua},
    pdfsubject={Cambridge Maths Notes: Part \npart\ - \ncourse\ (\nextra)},
    pdftitle={Part \npart\ - \ncourse\ (\nextra)},
  pdfkeywords={Cambridge Mathematics Maths Math \npart\ \nterm\ \nyear\ \ncourse\ \nextra}]{hyperref}

  \title{Part \npart\ --- \ncourse \\ {\Large \nextra}}
  \renewcommand\printindex{}
\fi

\pgfplotsset{compat=1.12}

\pagestyle{fancyplain}
\ifx \ncoursehead \undefined
\def\ncoursehead{\ncourse}
\fi

\lhead{\emph{\nouppercase{\leftmark}}}
\ifx \nextra \undefined
  \rhead{
    \ifnum\thepage=1
    \else
      \npart\ \ncoursehead
    \fi}
\else
  \rhead{
    \ifnum\thepage=1
    \else
      \npart\ \ncoursehead \ (\nextra)
    \fi}
\fi
\usetikzlibrary{arrows.meta}
\usetikzlibrary{decorations.markings}
\usetikzlibrary{decorations.pathmorphing}
\usetikzlibrary{positioning}
\usetikzlibrary{fadings}
\usetikzlibrary{intersections}
\usetikzlibrary{cd}

\newcommand*{\Cdot}{{\raisebox{-0.25ex}{\scalebox{1.5}{$\cdot$}}}}
\newcommand {\pd}[2][ ]{
  \ifx #1 { }
    \frac{\partial}{\partial #2}
  \else
    \frac{\partial^{#1}}{\partial #2^{#1}}
  \fi
}
\ifx \nhtml \undefined
\else
  \renewcommand\printindex{}
  \DisableLigatures[f]{family = *}
  \let\Contentsline\contentsline
  \renewcommand\contentsline[3]{\Contentsline{#1}{#2}{}}
  \renewcommand{\@dotsep}{10000}
  \newlength\currentparindent
  \setlength\currentparindent\parindent

  \newcommand\@minipagerestore{\setlength{\parindent}{\currentparindent}}
  \usepackage[active,tightpage,pdftex]{preview}
  \renewcommand{\PreviewBorder}{0.1cm}

  \newenvironment{stretchpage}%
  {\begin{preview}\begin{minipage}{\hsize}}%
    {\end{minipage}\end{preview}}
  \AtBeginDocument{\begin{stretchpage}}
  \AtEndDocument{\end{stretchpage}}

  \newcommand{\@@newpage}{\end{stretchpage}\begin{stretchpage}}

  \let\@real@section\section
  \renewcommand{\section}{\@@newpage\@real@section}
  \let\@real@subsection\subsection
  \renewcommand{\subsection}{\@ifstar{\@real@subsection*}{\@@newpage\@real@subsection}}
\fi
\ifx \ntrim \undefined
\else
  \usepackage{geometry}
  \geometry{
    papersize={379pt, 699pt},
    textwidth=345pt,
    textheight=596pt,
    left=17pt,
    top=54pt,
    right=17pt
  }
\fi

\ifx \nisofficial \undefined
\let\@real@maketitle\maketitle
\renewcommand{\maketitle}{\@real@maketitle\begin{center}\begin{minipage}[c]{0.9\textwidth}\centering\footnotesize These notes are not endorsed by the lecturers, and I have modified them (often significantly) after lectures. They are nowhere near accurate representations of what was actually lectured, and in particular, all errors are almost surely mine.\end{minipage}\end{center}}
\else
\fi

% Theorems
\theoremstyle{definition}
\newtheorem*{aim}{Aim}
\newtheorem*{axiom}{Axiom}
\newtheorem*{claim}{Claim}
\newtheorem*{cor}{Corollary}
\newtheorem*{conjecture}{Conjecture}
\newtheorem*{defi}{Definition}
\newtheorem*{eg}{Example}
\newtheorem*{ex}{Exercise}
\newtheorem*{fact}{Fact}
\newtheorem*{law}{Law}
\newtheorem*{lemma}{Lemma}
\newtheorem*{notation}{Notation}
\newtheorem*{prop}{Proposition}
\newtheorem*{question}{Question}
\newtheorem*{rrule}{Rule}
\newtheorem*{thm}{Theorem}
\newtheorem*{assumption}{Assumption}

\newtheorem*{remark}{Remark}
\newtheorem*{warning}{Warning}
\newtheorem*{exercise}{Exercise}

\newtheorem{nthm}{Theorem}[section]
\newtheorem{nlemma}[nthm]{Lemma}
\newtheorem{nprop}[nthm]{Proposition}
\newtheorem{ncor}[nthm]{Corollary}


\renewcommand{\labelitemi}{--}
\renewcommand{\labelitemii}{$\circ$}
\renewcommand{\labelenumi}{(\roman{*})}

\let\stdsection\section
\renewcommand\section{\newpage\stdsection}

\newcommand\qedsym{\hfill\ensuremath{\square}}
% Strike through
\def\st{\bgroup \ULdepth=-.55ex \ULset}


%%%%%%%%%%%%%%%%%%%%%%%%%
%%%%% Maths Symbols %%%%%
%%%%%%%%%%%%%%%%%%%%%%%%%

% Matrix groups
\newcommand{\GL}{\mathrm{GL}}
\newcommand{\Or}{\mathrm{O}}
\newcommand{\PGL}{\mathrm{PGL}}
\newcommand{\PSL}{\mathrm{PSL}}
\newcommand{\PSO}{\mathrm{PSO}}
\newcommand{\PSU}{\mathrm{PSU}}
\newcommand{\SL}{\mathrm{SL}}
\newcommand{\SO}{\mathrm{SO}}
\newcommand{\Spin}{\mathrm{Spin}}
\newcommand{\Sp}{\mathrm{Sp}}
\newcommand{\SU}{\mathrm{SU}}
\newcommand{\U}{\mathrm{U}}
\newcommand{\Mat}{\mathrm{Mat}}

% Matrix algebras
\newcommand{\gl}{\mathfrak{gl}}
\newcommand{\ort}{\mathfrak{o}}
\newcommand{\so}{\mathfrak{so}}
\newcommand{\su}{\mathfrak{su}}
\newcommand{\uu}{\mathfrak{u}}
\renewcommand{\sl}{\mathfrak{sl}}

% Special sets
\newcommand{\C}{\mathbb{C}}
\newcommand{\CP}{\mathbb{CP}}
\newcommand{\GG}{\mathbb{G}}
\newcommand{\N}{\mathbb{N}}
\newcommand{\Q}{\mathbb{Q}}
\newcommand{\R}{\mathbb{R}}
\newcommand{\RP}{\mathbb{RP}}
\newcommand{\T}{\mathbb{T}}
\newcommand{\Z}{\mathbb{Z}}
\renewcommand{\H}{\mathbb{H}}

% Brackets
\newcommand{\abs}[1]{\left\lvert #1\right\rvert}
\newcommand{\bket}[1]{\left\lvert #1\right\rangle}
\newcommand{\brak}[1]{\left\langle #1 \right\rvert}
\newcommand{\braket}[2]{\left\langle #1\middle\vert #2 \right\rangle}
\newcommand{\bra}{\langle}
\newcommand{\ket}{\rangle}
\newcommand{\norm}[1]{\left\lVert #1\right\rVert}
\newcommand{\normalorder}[1]{\mathop{:}\nolimits\!#1\!\mathop{:}\nolimits}
\newcommand{\tv}[1]{|#1|}
\renewcommand{\vec}[1]{\boldsymbol{\mathbf{#1}}}

% not-math
\newcommand{\bolds}[1]{{\bfseries #1}}
\newcommand{\cat}[1]{\mathsf{#1}}
\newcommand{\ph}{\,\cdot\,}
\newcommand{\term}[1]{\emph{#1}\index{#1}}
\newcommand{\phantomeq}{\hphantom{{}={}}}
% Probability
\DeclareMathOperator{\Bernoulli}{Bernoulli}
\DeclareMathOperator{\betaD}{beta}
\DeclareMathOperator{\bias}{bias}
\DeclareMathOperator{\binomial}{binomial}
\DeclareMathOperator{\corr}{corr}
\DeclareMathOperator{\cov}{cov}
\DeclareMathOperator{\gammaD}{gamma}
\DeclareMathOperator{\mse}{mse}
\DeclareMathOperator{\multinomial}{multinomial}
\DeclareMathOperator{\Poisson}{Poisson}
\DeclareMathOperator{\var}{var}
\newcommand{\E}{\mathbb{E}}
\newcommand{\Prob}{\mathbb{P}}

% Algebra
\DeclareMathOperator{\adj}{adj}
\DeclareMathOperator{\Ann}{Ann}
\DeclareMathOperator{\Aut}{Aut}
\DeclareMathOperator{\Char}{char}
\DeclareMathOperator{\disc}{disc}
\DeclareMathOperator{\dom}{dom}
\DeclareMathOperator{\fix}{fix}
\DeclareMathOperator{\Hom}{Hom}
\DeclareMathOperator{\id}{id}
\DeclareMathOperator{\image}{image}
\DeclareMathOperator{\im}{im}
\DeclareMathOperator{\tr}{tr}
\DeclareMathOperator{\Tr}{Tr}
\newcommand{\Bilin}{\mathrm{Bilin}}
\newcommand{\Frob}{\mathrm{Frob}}

% Others
\newcommand\ad{\mathrm{ad}}
\newcommand\Art{\mathrm{Art}}
\newcommand{\B}{\mathcal{B}}
\newcommand{\cU}{\mathcal{U}}
\newcommand{\Der}{\mathrm{Der}}
\newcommand{\D}{\mathrm{D}}
\newcommand{\dR}{\mathrm{dR}}
\newcommand{\exterior}{\mathchoice{{\textstyle\bigwedge}}{{\bigwedge}}{{\textstyle\wedge}}{{\scriptstyle\wedge}}}
\newcommand{\F}{\mathbb{F}}
\newcommand{\G}{\mathcal{G}}
\newcommand{\Gr}{\mathrm{Gr}}
\newcommand{\haut}{\mathrm{ht}}
\newcommand{\Hol}{\mathrm{Hol}}
\newcommand{\hol}{\mathfrak{hol}}
\newcommand{\Id}{\mathrm{Id}}
\newcommand{\lie}[1]{\mathfrak{#1}}
\newcommand{\op}{\mathrm{op}}
\newcommand{\Oc}{\mathcal{O}}
\newcommand{\pr}{\mathrm{pr}}
\newcommand{\Ps}{\mathcal{P}}
\newcommand{\pt}{\mathrm{pt}}
\newcommand{\qeq}{\mathrel{``{=}"}}
\newcommand{\Rs}{\mathcal{R}}
\newcommand{\Vect}{\mathrm{Vect}}
\newcommand{\wsto}{\stackrel{\mathrm{w}^*}{\to}}
\newcommand{\wt}{\mathrm{wt}}
\newcommand{\wto}{\stackrel{\mathrm{w}}{\to}}
\renewcommand{\d}{\mathrm{d}}
\renewcommand{\P}{\mathbb{P}}
%\renewcommand{\F}{\mathcal{F}}


\let\Im\relax
\let\Re\relax

\DeclareMathOperator{\area}{area}
\DeclareMathOperator{\card}{card}
\DeclareMathOperator{\ccl}{ccl}
\DeclareMathOperator{\ch}{ch}
\DeclareMathOperator{\cl}{cl}
\DeclareMathOperator{\cls}{\overline{\mathrm{span}}}
\DeclareMathOperator{\coker}{coker}
\DeclareMathOperator{\conv}{conv}
\DeclareMathOperator{\cosec}{cosec}
\DeclareMathOperator{\cosech}{cosech}
\DeclareMathOperator{\covol}{covol}
\DeclareMathOperator{\diag}{diag}
\DeclareMathOperator{\diam}{diam}
\DeclareMathOperator{\Diff}{Diff}
\DeclareMathOperator{\End}{End}
\DeclareMathOperator{\energy}{energy}
\DeclareMathOperator{\erfc}{erfc}
\DeclareMathOperator{\erf}{erf}
\DeclareMathOperator*{\esssup}{ess\,sup}
\DeclareMathOperator{\ev}{ev}
\DeclareMathOperator{\Ext}{Ext}
\DeclareMathOperator{\fst}{fst}
\DeclareMathOperator{\Fit}{Fit}
\DeclareMathOperator{\Frac}{Frac}
\DeclareMathOperator{\Gal}{Gal}
\DeclareMathOperator{\gr}{gr}
\DeclareMathOperator{\hcf}{hcf}
\DeclareMathOperator{\Im}{Im}
\DeclareMathOperator{\Ind}{Ind}
\DeclareMathOperator{\Int}{Int}
\DeclareMathOperator{\Isom}{Isom}
\DeclareMathOperator{\lcm}{lcm}
\DeclareMathOperator{\length}{length}
\DeclareMathOperator{\Lie}{Lie}
\DeclareMathOperator{\like}{like}
\DeclareMathOperator{\Lk}{Lk}
\DeclareMathOperator{\Maps}{Maps}
\DeclareMathOperator{\orb}{orb}
\DeclareMathOperator{\ord}{ord}
\DeclareMathOperator{\otp}{otp}
\DeclareMathOperator{\poly}{poly}
\DeclareMathOperator{\rank}{rank}
\DeclareMathOperator{\rel}{rel}
\DeclareMathOperator{\Rad}{Rad}
\DeclareMathOperator{\Re}{Re}
\DeclareMathOperator*{\res}{res}
\DeclareMathOperator{\Res}{Res}
\DeclareMathOperator{\Ric}{Ric}
\DeclareMathOperator{\rk}{rk}
\DeclareMathOperator{\Rees}{Rees}
\DeclareMathOperator{\Root}{Root}
\DeclareMathOperator{\sech}{sech}
\DeclareMathOperator{\sgn}{sgn}
\DeclareMathOperator{\snd}{snd}
\DeclareMathOperator{\Spec}{Spec}
\DeclareMathOperator{\spn}{span}
\DeclareMathOperator{\stab}{stab}
\DeclareMathOperator{\St}{St}
\DeclareMathOperator{\supp}{supp}
\DeclareMathOperator{\Syl}{Syl}
\DeclareMathOperator{\Sym}{Sym}
\DeclareMathOperator{\vol}{vol}

\pgfarrowsdeclarecombine{twolatex'}{twolatex'}{latex'}{latex'}{latex'}{latex'}
\tikzset{->/.style = {decoration={markings,
                                  mark=at position 1 with {\arrow[scale=2]{latex'}}},
                      postaction={decorate}}}
\tikzset{<-/.style = {decoration={markings,
                                  mark=at position 0 with {\arrowreversed[scale=2]{latex'}}},
                      postaction={decorate}}}
\tikzset{<->/.style = {decoration={markings,
                                   mark=at position 0 with {\arrowreversed[scale=2]{latex'}},
                                   mark=at position 1 with {\arrow[scale=2]{latex'}}},
                       postaction={decorate}}}
\tikzset{->-/.style = {decoration={markings,
                                   mark=at position #1 with {\arrow[scale=2]{latex'}}},
                       postaction={decorate}}}
\tikzset{-<-/.style = {decoration={markings,
                                   mark=at position #1 with {\arrowreversed[scale=2]{latex'}}},
                       postaction={decorate}}}
\tikzset{->>/.style = {decoration={markings,
                                  mark=at position 1 with {\arrow[scale=2]{latex'}}},
                      postaction={decorate}}}
\tikzset{<<-/.style = {decoration={markings,
                                  mark=at position 0 with {\arrowreversed[scale=2]{twolatex'}}},
                      postaction={decorate}}}
\tikzset{<<->>/.style = {decoration={markings,
                                   mark=at position 0 with {\arrowreversed[scale=2]{twolatex'}},
                                   mark=at position 1 with {\arrow[scale=2]{twolatex'}}},
                       postaction={decorate}}}
\tikzset{->>-/.style = {decoration={markings,
                                   mark=at position #1 with {\arrow[scale=2]{twolatex'}}},
                       postaction={decorate}}}
\tikzset{-<<-/.style = {decoration={markings,
                                   mark=at position #1 with {\arrowreversed[scale=2]{twolatex'}}},
                       postaction={decorate}}}

\tikzset{circ/.style = {fill, circle, inner sep = 0, minimum size = 3}}
\tikzset{scirc/.style = {fill, circle, inner sep = 0, minimum size = 1.5}}
\tikzset{mstate/.style={circle, draw, blue, text=black, minimum width=0.7cm}}

\tikzset{eqpic/.style={baseline={([yshift=-.5ex]current bounding box.center)}}}
\tikzset{commutative diagrams/.cd,cdmap/.style={/tikz/column 1/.append style={anchor=base east},/tikz/column 2/.append style={anchor=base west},row sep=tiny}}

\definecolor{mblue}{rgb}{0.2, 0.3, 0.8}
\definecolor{morange}{rgb}{1, 0.5, 0}
\definecolor{mgreen}{rgb}{0.1, 0.4, 0.2}
\definecolor{mred}{rgb}{0.5, 0, 0}

\def\drawcirculararc(#1,#2)(#3,#4)(#5,#6){%
    \pgfmathsetmacro\cA{(#1*#1+#2*#2-#3*#3-#4*#4)/2}%
    \pgfmathsetmacro\cB{(#1*#1+#2*#2-#5*#5-#6*#6)/2}%
    \pgfmathsetmacro\cy{(\cB*(#1-#3)-\cA*(#1-#5))/%
                        ((#2-#6)*(#1-#3)-(#2-#4)*(#1-#5))}%
    \pgfmathsetmacro\cx{(\cA-\cy*(#2-#4))/(#1-#3)}%
    \pgfmathsetmacro\cr{sqrt((#1-\cx)*(#1-\cx)+(#2-\cy)*(#2-\cy))}%
    \pgfmathsetmacro\cA{atan2(#2-\cy,#1-\cx)}%
    \pgfmathsetmacro\cB{atan2(#6-\cy,#5-\cx)}%
    \pgfmathparse{\cB<\cA}%
    \ifnum\pgfmathresult=1
        \pgfmathsetmacro\cB{\cB+360}%
    \fi
    \draw (#1,#2) arc (\cA:\cB:\cr);%
}
\newcommand\getCoord[3]{\newdimen{#1}\newdimen{#2}\pgfextractx{#1}{\pgfpointanchor{#3}{center}}\pgfextracty{#2}{\pgfpointanchor{#3}{center}}}

\newcommand\qedshift{\vspace{-17pt}}
\newcommand\fakeqed{\pushQED{\qed}\qedhere}

\def\Xint#1{\mathchoice
   {\XXint\displaystyle\textstyle{#1}}%
   {\XXint\textstyle\scriptstyle{#1}}%
   {\XXint\scriptstyle\scriptscriptstyle{#1}}%
   {\XXint\scriptscriptstyle\scriptscriptstyle{#1}}%
   \!\int}
\def\XXint#1#2#3{{\setbox0=\hbox{$#1{#2#3}{\int}$}
     \vcenter{\hbox{$#2#3$}}\kern-.5\wd0}}
\def\ddashint{\Xint=}
\def\dashint{\Xint-}

\newcommand\separator{{\centering\rule{2cm}{0.2pt}\vspace{2pt}\par}}

\newenvironment{own}{\color{gray!70!black}}{}

\newcommand\makecenter[1]{\raisebox{-0.5\height}{#1}}

\mathchardef\mdash="2D

\newenvironment{significant}{\begin{center}\begin{minipage}{0.9\textwidth}\centering\em}{\end{minipage}\end{center}}
\DeclareRobustCommand{\rvdots}{%
  \vbox{
    \baselineskip4\p@\lineskiplimit\z@
    \kern-\p@
    \hbox{.}\hbox{.}\hbox{.}
  }}
\DeclareRobustCommand\tph[3]{{\texorpdfstring{#1}{#2}}}
\makeatother


\begin{document}

\maketitle


\tableofcontents


\setcounter{section}{-1}
\section{Introduction}


\section{Physics on the go}
\subsection{Mechanics}

\begin{defi}[Velocity]
  The \emph{velocity} of the particle is
  \[
    \mathbf{v} = \dot{\mathbf{r}} = \frac{d \mathbf{r}}{d t}.
  \]
\end{defi}

\begin{defi}[Acceleration]
  The \emph{acceleration} of the particle is
  \[
    \mathbf{a} = \dot{\mathbf{v}} = \ddot{\mathbf{r}} = \frac{d^2 \mathbf{r}}{d t^2}.
  \]
\end{defi}

\begin{defi}[Momentum]
  The \emph{momentum} of a particle is
  \[
    \mathbf{p} = m\mathbf{v} = m\dot{\mathbf{r}}.
  \]
  In particular , \emph{Momentum} is conserved in \emph{a closed system}
  
  $m$ is the \emph{inertial mass} of the particle, and measures its reluctance to accelerate, as described by Newton's second law.
\end{defi}

\begin{law}[Newton's First Law of Motion]
  A body remains at rest, or moves uniformly in a straight line, unless acted on by a force. (This is in fact Galileo's Law of Inertia)
\end{law}

\begin{law}[Newton's Second Law of Motion]
   The rate of change of momentum of a body is equal to the \emph{Resultant force} acting on it (in both magnitude and direction).
\end{law}

\begin{law}[Newton's Third Law of Motion]
  To every action there is an equal and opposite reaction: the forces of two bodies on each other are equal and in opposite directions.
\end{law}

\begin{defi}[Resultant Force]
combine Force vector , If Resultant Force =0 , the object will be constant velocity or at rest.
\end{defi}

\begin{defi}[Force in equilibrium]
  If you have a free-body diagram , you can combine the forces with a vector diagram . If this produces a closed vector polygon , the resultant is zero . This means they are in \emph{equilibrium}.
 
  
\end{defi}

\begin{defi}[Acceleration of free fall and Weight]

  
\end{defi}


\begin{defi}[Center of Mass/gravity]
  The position through which all the weight acts on it if it will \emph{balance in a pivot}.

\begin{defi}[Projectile motion]
\begin{enumerate}
    \item h independent with v
    \item constant h v as air resistance
    \item downward force is acting (gravity)
\end{enumerate}
\end{defi}

\end{defi}
\begin{defi}[Work Done]
  The Energy transfer by Forces with direction.
\end{defi}

\subsection{Material}
\begin{defi}[Density]

\end{defi}
\begin{defi}[Upthrust]
  The Upthrust is upward force when a body is fully or partially submerged in a liquid , pressure differences at different depths.
\end{defi}

\begin{defi}[Laminar flow and Turbulent flow]
Laminar flow is not chaotic (continuous lines) and occurs at lower speeds and no mixing of layers , so it's not include eddies, subject to sudden changes in speed.\\

Turbulent flow is chaotic and subject to sudden changes in speed and direction - eddies are frequently seen . There is a lot of large-scale mixing of layers .
\end{defi}

\begin{defi}[Viscous drag]
The fictional force act when solids and fluids move relative to each other , the layer of fluid next to the solid exerts a friction force on it.
\end{defi}
\begin{defi}[Viscosity]
Viscous drag would be greater in syrup than in water. We say the syrup has a greater viscosity , or that is more viscous.\

The coefficient of viscosity ,$\eta$ , is the value to measure the viscosity.
\end{defi}

\begin{prop}
The viscosity inversely proportional to temperature.
\end{prop}



\begin{law}[Stokes' law]
  \begin{equation*}
      F=6\pi\eta rv
  \end{equation*}
  where F is viscous drag force
\end{law}
\begin{defi}[Terminal velocity]
If an object are falling through a fluid, The resultant force is
\begin{equation*}
    upthrust+viscous drag=weight
\end{equation*}
Upthrust and weight are constant and the viscous drag is proportional to the downward velocity of the object\\

As the resultant force is zero , there is no more acceleration. The velocity is constant at which this occurs is known as the terminal velocity by N1.
\end{defi}

\begin{prop}
The terminal velocity can be expressed in terms of Stokes' \begin{equation*}
    \frac{4}{3}\pi r^3\rho_{fluid}g+6\pi\mu rv=\frac{4}{3}\pi r^3\rho_{steel}g
\end{equation*}
Rearranging this further gives
\begin{equation*}
    v=\frac{2r^2g(p_{steel}-p_{fluid}}{9\mu}
\end{equation*}
\end{prop}

\begin{law}[Hooke's law]
The extension proportional to Applied Force, The Force cannot exceed before the limit of proportionality.\\
k is called stiffness or spring constant\\

extension and compression define as same way\\

If to spring are in series, Both string have the same applied force and shares half of total extension, If string are in parallel , both spring give the same extension and shares half of appiled force. 
\end{law}

\begin{defi}[Stress and Strain graphs]
P,E(the maximum extent to which a solid may be stretched without permanent alternation of size or slip),Y(become plastic)
and highest point of curve is breaking stress(UTS Ultimate tensile strength)

Breaking stress is the maximum stress that a material can withstand without breaking.If the applied stress is just equal to this , the material would break

the graph may have permanent strain in x-aixs which is non-zero
\end{defi}



\begin{defi}[Young modulus]
The gradient of stress-strain curve is Young modulus\
If gradient of $A > B$ , A is stiffer then B\\
given by
\begin{equation*}
    \textbf{Young modulus}=\frac{\textbf{stress}}{\textbf{strain}}=\frac{\textbf{applied force}/\textbf{cross-sectional area}}{\textbf{extension}/\textbf{orginal length}}
    \\=\frac{\textbf{applied force}\times \textbf{original length}}{\textbf{cross-sectional area}\times \textbf{extension}}
\end{equation*}

\end{defi}
\begin{defi}[Graph]
stress-strain and Force-Extension\\

If the load is too high , the elastic limit of the spring will be exceeded , and the spring will not return to its original length/position.\\

High UTS: Can withstand large stress / force / tension\\

Higher elastic limit so will return to its original length/shape if greater forces are applied\\

higher ultimate stress so stronger so the thread could be thinner for same (cross-sectional) area can with stand larger force\\

Larger area under the graph so tougher and can absorb more energy\\

Larger gradient and greater Young modulus for the same stress/force so stiffer.
\end{defi}



\begin{defi}[Elastic strain energy]
Work is Done by the deforming force in extending it , and the energy is stored in the spring as elastic strain energy, which released when the force is removed. The force is not constant
\begin{equation*}
    E_{el}=\frac{1}{2}F\delta x=\frac{1}{2}k\delta x^2
\end{equation*}
\end{defi}



\begin{defi}[Type of deformation]
Elastic and Plastic can return or not to its original shape  and There is nor permanent deformation when stress is removed.
\end{defi}


\begin{defi}[Describing materials]
\end{defi}
\begin{center}
\begin{tabular}{ |l| c| r| } \hline Property & Behavior  \\ \hline Hight UTS or strong or brittle & Will not break when opened/when force/stress applied  \\ \hline High Young Modulus or stiff & Grips paper (firmly)   \\ \hline Ductile & Can be drawn into wires\\ \hline Malleable/ductile & Can be bent into shape\\ \hline Elastic & Will close after being opened \\ \hline \end{tabular}
\end{center}

\begin{remark}
The opposite word of stiff is flexible(Stiffness:is tendency to resist deformation by a tensile force) with low YM, and also strong and weak with low UTS. Ductility(indicate the type of deformation of a material before breaking) can be said by brittle(glass) and ductile(metal) with no and high plastic deformation respectively.
\end{remark}

\begin{ex}[Example of metal]
\end{ex}

\begin{center}
\begin{tabular}{ |l| c| r| } \hline & strength & Stiffness   \\ \hline Steel & strong & stiff \\ \hline Chalk & weak & stiff  \\ \hline nylon & strong & flexible \\ \hline jelly & weak & flexible \\ \hline  \end{tabular}
\end{center}

\begin{defi}[Hardness]
A material is hard if it can resist plastic deformation by scratching \\

descending order to hardness:diamond, ruby(red) and sapphire(blue), steel 
\end{defi}

\begin{defi}[Toughness]
A material is though if it can withstand shock and impact without breaking/if it can absorb a lot of energy in the plastic region \\

long strain with stort stress in the curve is also though
\end{defi}


\section{Physics at work}
\subsection{Wave}
\begin{defi}[Property of Wave]
\begin{enumerate}
    \item amplitude--maximum point of displacement
    \item wavelength $\lambda$The distance of a wave travelled one complete oscillation
    \item frequency $f$The number of oscillation of a wave in one second
    \item period $T$  $T=\frac{1}{f}$
    \item speed $v$
\end{enumerate}

\begin{defi}[Transverse wave]
The Oscillation are at perpendicular to the direction of travel of the wave.
\end{defi}

\begin{defi}[longitudinal wave]
The vibration in a longitudinal wave is parallel to the direction of travel of the wave
\end{defi}

\begin{defi}[wavefront]
is a line representing a series of equivalent points on the waves (eg.all the compressions)
\end{defi}

\begin{defi}[wave graph]
displacement-time / displacement-distance

sd:period $\rightarrow$ wavelength\\
st:period $\rightarrow$ period
\end{defi}

\end{defi}

\begin{defi}[Diffraction]
most when wavelength approximately equal to gap size.\\

High Frequency gives a shorter wavelength so there is less diffraction by $v=f\lambda$\\


\end{defi}
\begin{defi}[Diffraction Pattern]
Electron spread out form a diffraction/ interference pattern or undergo superposition.\\
Electrons must behave as waves
//
Or electrons have a wavelength, similar to the atomic spacing\\
Because diffraction/interference is a wave behaviour
\end{defi}

\begin{defi}[sound wave]
longitudinal wave, Particles of air are displaced from their equilibrium position and produce regions of compression and rarefaction
\end{defi}

\begin{defi}[Superposition]
When two wave or more waves arrive at the same position and same time , superposition take place.\\

constructive/in phase
destructive/out phase
\end{defi}

\begin{defi}[Interference]
Two sets of wave with the \emph{same frequency} and a constant phase difference are said to be coherent and produce pattern
\end{defi}

\begin{defi}[phase difference and path difference]
path difference=$\lambda_1-\lambda_2$
If the answer is integer of $\lambda$ , that is constructive/in-phase , otherwise , that is destructive / out of phase

Phase difference=$\Delta\theta$
\end{defi}

\begin{defi}[Standing wave]
Superposition of a continuous reflected from a fixed end with its incident wave will produce an interference pattern called Standing wave or stationary wave , The maxima(of amplitude) are called anti-nodes and the minima called nodes.
\begin{equation*}
    L=\frac{n}{2}\dot \lambda
\end{equation*}
\end{defi}

\begin{defi}[refraction]
When waves meet a boundary between two materials , some of the wave is reflected and some is transmitted . The transmitted wave changes speed and may change direction.
\end{defi}

\begin{defi}[Refractive index]

Refractive index define by
\begin{equation*}
    _1\mu_2=\frac{v_1}{v_2}
\end{equation*}
or
\begin{equation*}
    _1\mu_2=\frac{\sin i}{\sin r}
\end{equation*}
\end{defi}

\begin{defi}[total internal reflection]
If a wave passes from a more dense to a less dense material , it is possible for all the light to reflect and none to refract at that interface . This is called total internal reflection . This happens if the angle of incidence within the material is greater then the critical angle CAngle of incidence (for light travelling from denser medium) Has angle of refraction of 90  (may refer to leaving along surface/boundary). At smaller angles , some of the light may be reflected\\

The critical angle is related to the refractive index. Note that , when the angle in the material$=C$ then $_1\mu_2=\sin 90/\sin C$ and as $\sin 90=1$ , this simplifies to:
\begin{equation*}
    _1\mu_2=\frac{1}{\sin C}
\end{equation*}
\end{defi}

\begin{defi}[Doppler effect]
When waves are emitted from a moving source or detected by a moving receiver , the detected frequency differs from the emitted frequency.The shift in frequency is proportional to the relative speed  of the motion.If source moving toward to detector higher f/pitch and wavelength shorter, otherwise opposite.

If the source were moving faster,the detector detect would be different as change of f is greater
\end{defi}


\begin{ex}[Ultrasound scanning]
An Ultrasound scan of a foetus is usually taken at about 12 weeks. Reflected pules of ultrasound are to determine where the boundaries are between different tissues and then build an image
\end{ex}

\begin{defi}[revolution]
The revolution (The smallest level of detail that can be seen) can be improved by reducing the wavelength $\lambda$ of the sound used. Revolution can also improved by using pules of very short time interval.\\

Shorter pulse duration:
Shorter pulses have a shorter length so allow great detail.
\end{defi}

\begin{defi}[Separated by a shorter time interval]
Separated by a shorter time so the pulse travel a smaller distance and they return more quickly so to allow more frequent monitoring of the objects
\end{defi}

\begin{defi}[Polarization]
Plane of polarization of lens must be aligned at 90 degrees to plane to polarization of reflected light.
\end{defi}


\begin{defi}[light]
Light is a transverses wave . It consists of varying electric and magnetic fields at right angles to its direction of motion
\begin{enumerate}
    \item In unpolarised light, these variations take place in ALL planes(many direction) at right angles to the direction in which the ray of light is travelling.
    \item In plane polarised light , the variations in electric field take place only in ONE plane. The variations in magnetic field (propagation) are in a plane at right angles to this
\end{enumerate}
Longitudinal waves such as sound cannot be polarised
\end{defi}

\begin{defi}[Polarising filters]
If unpolarised light encounters a polarising filter , some of it is absorbed and the emerging light is polarised.\\

If polarised light encounters a polarizing filter, polarised light emerges , and its brightness and plane of polarisation depend on the orientation of the filter.\\

If two polarising filters are arranged so that they are orientated at right angles to each other , then they will completely absorb unpolarised light . The filters are said to be crossed. 
\end{defi}

\begin{defi}[Optical activity]
Optically active substances such as sugar solutions rotate the plane of polarisation by an amount proportional to their concentration and the depth of liquid through which the light travels. This can be used to measure the concentration of sugar solutions\\.

In practically,\\
In Optics,
\begin{enumerate}
    \item There is a change in density from water to air
    \item Or there is a change in light speed from water to air
    
\end{enumerate}
This causes a change in direction of light moving away from normal travelling from water to air.So light appears to come from a different point of origin.
\end{defi}




\subsection{DC electricity}
\begin{defi}[Current]
\begin{equation*}
    I=\frac{\Delta Q}{\Delta t}
\end{equation*}
The SI unit of electric charge is the coulomb ,C , and the SI unit of current is the ampere, A , then:
\begin{equation*}
    1 A= 1Cs^{-1}
\end{equation*}
The current start at positive charge, the flow of electrons is negative charge
\end{defi}
\begin{defi}[Potential difference]
\begin{equation*}
    V=\frac{W}{Q}
\end{equation*}
where W is the work done (The energy transfer)\\

The SI unit of potential difference is the volt, V
\begin{equation*}
    1 V=1JC^{-1}
\end{equation*}
\end{defi}

\begin{defi}[Emf]
Electromotive force is a measure applied to a source of electrical power such as a battery . It is the energy available per coulomb of charge.
\end{defi}

\begin{defi}[Resistance]
\begin{equation*}
    V=IR
\end{equation*}
\end{defi}

\begin{law}[Ohm's law]
For metals at a constant temperature , the current in the metal is proportional to the voltage across it.
\end{law}

\begin{defi}[Resistance and temperature]
The resistance of many electrical components change with temperature . A current flowing through a component can raise its temperature and so change its resistance.
\end{defi}

\begin{defi}[Metal]

\end{defi}
\begin{defi}[Semiconductors]

\end{defi}
\begin{defi}[Power]
\begin{equation*}
    P=IV=\frac{V^2}{R}
\end{equation*}
Power depend with brightness\\

Power imply more electron per second and more energy loss for each electron
\end{defi}
\begin{defi}[Resistors in series or parallel]

\end{defi}
\begin{defi}[Resistivity]
Resistivity of conductor is:
\begin{equation*}
    R=\frac{pl}{A}
\end{equation*}
where l is length of conductor and A is its cross-sectional area.
\end{defi}

\begin{defi}[Drift velocity]
Drift velocity (of electrons) increases (as current increases)\\
More (frequent) collisions of electrons with lattice ions\\
More energy(T increase) transferred when electrons collide with lattice Ions then low drift velocity
\end{defi}
You can do the explanation of the increases of resistance of metal when T increase

\begin{defi}[semiconductor]
If the T increase , the number of charge carries increase as It include silicon and germanium. Then current increases (R decreases)
\end{defi}

\begin{defi}[Number density of conduction electrons]
The number density n is the number of conduction (free)
 electrons per unit volume of a material as A material with a small value of resistivity will be a good conductor and will have a large number of free electrons.
\begin{equation*}
    n=\frac{I}{qvA}
\end{equation*}
where I is current in a sample. q is the charge on an electron , v is the drift velocity of the free electrons , and A is the cross-sectional area of the sample.
\end{defi}

\begin{defi}[Ammeter and Voltmeter]
Voltmeter (high resistance) so very little/ negligible/zero current in the voltmeter (otherwise the current is flowed it), because that would change / increase the current in the ammeter

Ammeter (should be low resistance) so the current flow it. otherwise the current decrease but doesn't affect the calculation as current through the r is measured.
\end{defi}

\begin{defi}[The potential divider]
If a set of parallel, they have same volt\\
Voltage is zero on the end\\
the current is constant in the start and end.\\
current will proportional to flow in the parallel

using resistance wire
\end{defi}

\begin{defi}[Internal resistance]
\begin{equation*}
    emf=IR+Ir
\end{equation*}
that means
\begin{equation*}
    Energy supplied to each coulomb =energy transferred by load resistor+energy transferred due to internal heating
\end{equation*}
\end{defi}
\begin{defi}[Terminal potential difference and lost volts]
\begin{equation*}
    V=Ir
\end{equation*}
where r is internal Resistance. Ir is sometimes known as the lost volts.
\begin{enumerate}
    \item open circuit means that the power supply has no connection between its terminals, or is connected to a very high resistance. eg. a voltmeter. Then I=0 and V=emf
    \item short circuit means that the power supply terminals are joined by a connection with no Resistance . Then V=0 and I=emf/r.
\end{enumerate}
\end{defi}
\subsection{Nature of light}
\begin{defi}[Work function]

\end{defi}

\begin{defi}[Photon]
Packet/package/quantum of electromagnetic energy

electron :Energy like atom and behave like wave
\end{defi}

\begin{defi}[Photo-electric effect/emission]
Photons from incident light cause emission of electrons from surface of metal, and photon has energy $E=hf$ . There is emission only if photon energy greater than or equal to Work function $\Phi$\\

$\Phi$ is the minimum energy required for emission of electrons(Photo-electron) , $\frac{1}{2}mv^2$ is the kinetic energy of the emitted electrons . It is max because some energy may be transferred to the metal

photon energy$=$work done in releasing the electron $+$ kinetic energy of electron\\
if photon energy greater than or equal to Work function $\Phi$\\
\end{defi}

\begin{defi}[Stopping Voltage]
If the metal surface is connected to a positive potential , the photo-electron is attracted back to it . To escape from the surface the kinetic energy of the photo-electron is used to do work against the electrostatic force . If the potential is increased , eventually even the most energetic electrons fail to escape, and the potential is called the stopping voltage $V_s$ . The charge on each electron is $e$.
\begin{align*}
    \frac{1}{2}mv^2_{max}=eV_s\\
    hf=\phi+eV_s\\
    hf=hf_0+eV_s
\end{align*}
This can be rearranged in the form of the general equation of a straight line , $y=mx+c$
\begin{equation*}
    V_s=(\frac{h}{e})f-(\frac{hf_0}{e})
\end{equation*}
So a graph of $V_s$(y-axis) against f(Photon frequency)(x-axis) will have gradient $hle$ and intercept $-hf_0le$
\end{defi}

\begin{defi}[Energy Level]
The photon energies can be calculated from measurements of the wavelengths emitted or absorbed (emission and absorption Spectra). An energy level diagram is like a graph with energy increasing upwards.A downward-pointing arrow represents and electron losing energy and giving out a photon that is energy change of atom
\begin{equation*}
    hf=\Delta E
\end{equation*}
Max energy change, highest level change to ground state.
\end{defi}

\begin{defi}[Duality]
Wave Theory of Light:
\begin{enumerate}
    \item Diffraction occurs and fewer electrons emitted
    \item Lower intensity of waves would provide less energy to release fewer electrons . Lower intensity would mean a longer time for sufficient energy to be absorbed for electron release.
\end{enumerate}
Particle Theory of Light
\begin{enumerate}
    \item Fewer photons would release fewer electrons , write down photo-electric effect.
\end{enumerate}
\end{defi}
\begin{defi}[Radiation flux]
\begin{equation*}
    F=\frac{P}{A}
\end{equation*}
F is Flux $Wm^{-2}$, and $P=\frac{E}{t}$ is the power and A is the area PERPENDICULAR to the beam
\end{defi}

Solar cell

\section{Physics on the move}
\subsection{Further Mechanics}
\begin{defi}[Momentum]
  The \emph{momentum} of a particle is
  \[
    \mathbf{p} = m\mathbf{v} = m\dot{\mathbf{r}}.
  \]
  In particular , \emph{Momentum} is conserved in \emph{a closed system}
  
  $m$ is the \emph{inertial mass} of the particle, and measures its reluctance to accelerate, as described by Newton's second law.
\end{defi}

\begin{law}[Newton's Second Law of Motion]
   The rate of change of momentum of a body is equal to the \emph{Resultant force} acting on it (in both magnitude and direction).
   \begin{equation*}
       F=\frac{\Delta(mv)}{\Delta t}
   \end{equation*}
   where F is the resultant force and $\Delta t$ is the time for which the force acts.\\
   $\Delta$ delta means "change in"
\end{law}

\begin{defi}[Impulse]
Re-arranging the equation from Newton;s second law gives:
\begin{equation*}
    F \times \Delta t= \Delta (mv)
\end{equation*}
where $\Delta t$ is Impact time\\
The product of force and time for which the force acts is sometimes referred to as the impulse.That is equal to the change in momentum.\\

a resultant force must act on the object. The longer time for which the force acts, the smaller the force needed for a given change in momentum.

\end{defi}

\begin{defi}[Conservation of momentum]
In any collision in which no external force act , the total momentum remains constant.
\begin{equation*}
    m_1u_1+m_2u_2=m_1v_1+m_2v_2
\end{equation*}
where 1 and 2 refer to the situation just before and just after the collision
\end{defi}

\begin{defi}[Type of Collision]
Fixed target\\
There is momentum before the collision so there must be momentum after the collision. So particles created must have some kinetic energy. SO not all KE converted to mass\\

Colliding beams\\
(If particles have the same mass and speed) total initial momentum is zero, momentum after collision will be zero,\\

If one stationary particle is created, All of the kinetic energy of the particle is converted to mass\\

Explosion:\\
Under explosion , the momentum us still the same as the momentum before collision, So in order to conserve momentum , the products must be travelling different directions.
\end{defi}

\begin{defi}[Energy in collisions]
In elastic collisions the total kinetic energy $E_k$ is conserved . otherwise is referred to as an inelastic collision.The energy has become some other form such as thermal energy. When Objects STICK together a large fraction of the initial kinetic energy is usually converted to other forms.\\
Note that , since $p=mv$
\begin{align*}
    p^2=m^2v^2 \Rightarrow \frac{p^2}{2m}=\frac{mv}{2}=E_k\\
    E_k=\frac{p^2}{2m}
\end{align*}
\end{defi}

\begin{defi}[Oblique (glancing) impacts]
If we resolve two perpendicular directions of momentum
\begin{align*}
    \frac{p^2}{2m}=\frac{p_1^2}{2m}+\frac{p_2^2}{2m}
\end{align*}
For momentum conservation ($p=p_1+p_2$) the object must move at right angles to each other after the collision, otherwise the two equations cannot be consistent.\\

We can also take horizontally and vertically to resolve this type of questions.
\end{defi}

\begin{defi}[Angular velocity]
\begin{equation*}
    v=\frac{2\pi r}{T}
\end{equation*}
where T is time taken for one complete revolution\\

We would consider the angular displacement $\Delta\theta$ that is measured in radians. The angle $\theta$ defined as $\theta=l/r$ where l is the arc length that gives the angle $\theta$ at the centre of the circle of radius r . Since 2$\pi$ radians make a complete circle (360) to convert an angle in degrees into radians you divide the angle by 360 and multiply by $2\pi$\\

The angular velocity $\omega$ is given by:
\begin{equation*}
    \omega=\frac{\Delta\theta}{\Delta t}
\end{equation*}
Units rad $s^{-1}$\\

The tangential velocity and the time for one revolution can both be expressed in terms of angular velocity
\begin{equation*}
    v=\omega r \mbox{~and~} T=\frac{2\pi}{\omega}
\end{equation*}

\end{defi}

\begin{defi}[Centripetal acceleration]
When an object moves in a circular path it is changing its direction and therefore it is accelerating. If the object is moving with a constant speed then the acceleration must be directed towards the centre of the circular path, perpendicular to its tangential motion.
\begin{equation*}
    a=\frac{v^2}{r}
\end{equation*}
Since $v=\omega r$
\begin{equation*}
    a=\frac{(\omega r)^2}{r}=\omega^2r
\end{equation*}
Similarly , Centripetal Force is multiply mass of an object 
\begin{equation*}
    F=\frac{mv^2}{r}
\end{equation*}

The centripetal force must arise from external forces that act on the object . For example , If a ball on a string is whirled in a horizontal circle the centripetal force is provided by the tension in the string.\\

You can using conservation of energy to resolve the angular motion question.

\end{defi}

\begin{ex}[An object is banking on Circular motion]
Consider An object has Weight mg , is banking on Circular motion and has an angle $\theta$,and the centripetal force L, we can consider about vertical force and horizontal force, we have to find the angle $\theta$\\
Apply Newton ' first law to vertical direction:
\begin{equation*}
    L\cos\theta=mg
\end{equation*}
Apply Newton ' second law to horizontal direction:
\begin{equation*}
    L\sin\theta=m\frac{v^2}{r}
\end{equation*}
\begin{align*}
    \frac{L\sin\theta}{L\cos\theta}=\frac{\frac{mv^2}{r}}{mg}\\
    \Rightarrow \tan\theta\frac{v^2}{gr}
\end{align*}
\end{ex}




\subsection{Electric and magnetic fields}

\begin{defi}[Coulomb's law]
The electrostatic force between two charged spherical objects obeys an inverse square law
\begin{equation*}
    F=\frac{kQ_1Q_2}{r^2}
\end{equation*}
if the charges is opposite then the force is attractive and negative.otherwise the charges is same then the force is repulsive and positive. The magnitude of both type of force is same
\end{defi}

\begin{defi}[Radial electric fields]
There is an electric field around every charged object. The interaction between electric fields produces electrostatic forces\\
the electric field strength is E, please see below.
\end{defi}

\begin{defi}[Uniform electric field]
Space/area/region where a force acts on a charged particle, The force is the same at all points.
$+\rightarrow-$
the Electric fields strength defined by:
\begin{equation*}
    E=\frac{F}{Q}
\end{equation*}
E has unit $NC^{-1}$ and that is vector\\
Can be used to accelerate/deflect particles.\\
Direction of force indicates (sign of) charge.
\begin{equation*}
    a=\frac{EQ}{m}
\end{equation*}
\end{defi}

\begin{defi}[Capacitors]
A capacitor is a device that stores energy by separating charge. Placing charge on the capacitor results in a potential difference. The capacitance C, of the system is defined as the charge stored per unit p.d. V.
\begin{equation*}
    C=\frac{Q}{V}
\end{equation*}
Note that the unit is F= $CV^{-1}$
\end{defi}

\begin{defi}[Charging and Discharging]
A potential difference builds up across a capacitor as it charges. The capacitor is fully charged once the potential difference across it becomes equal to the emf of the source. If we disconnect the capacitor from the source.we can use it to make a current flow around a circuit.
\end{defi}

\begin{defi}[Energy store in capacitor]
Energy is stored in the capacitor because work is done as charge moves through the net potential difference in the circle. This becomes electrostatic potential energy (i.e. energy stored in the electric field between the capacitor plates)\\

The Work done W, is equal to the shaded area under the graph of potential difference between the plates V, against charge on the plates Q.
\begin{equation*}
    W=QV_{av}
\end{equation*}
where V is the average potential difference that the charge moves through. Since there is a proportional relationship between Q and V , the average potential difference is $V/2$, where V is the maximum p.d. Hence the energy stored is given by:
\begin{equation*}
    W=Q\times \frac{1}{2}V
\end{equation*}
Since$Q=CV$
\begin{equation*}
    W=\frac{1}{2}CV^2
\end{equation*}
\end{defi}

\begin{defi}[Discharge of a capacitor]
The discharging process is an example of exponential decay , or constant ratio change. The time taken for the charge on the capacitor to fall to a given fraction of its staring value is always the same for a given circuit. This depends upon the capacitance, C , and resistance , R , in the circuit.
\begin{equation*}
    Q=Q_0e^{\frac{-t}{RC}}
\end{equation*}
where $Q_0$ is the initial charge on the capacitor, Q is the charge remaining on the capacitor after a time t.\\
the cases also can apply on current-time or Voltage-time relationship
\begin{equation*}
    I(V)=I(V)_0e^{\frac{-t}{RC}}
\end{equation*}
The charge of capacitor define as
\begin{equation*}
    stuff=stuff_0-stuff_0e^{\frac{t}{RC}}
\end{equation*}
\end{defi}
\begin{defi}[Time Constant]
If t=RC is put into equation, The $Q=Q_0e^{-1}$, then$\frac{Q}{Q_0}=\frac{1}{e}=0.37$
time constant is the time taken for the charge on a capacitor Q to fall to 37 percent of $Q_0$.It's also the time taken for the charge of a charging capacitor to rise to 63 percent of $Q_0$.The larger the resistance in series with the capacitor , the longer it takes to charge of discharge. In practice, The time taken for a capacitor to charge or discharge fully is taken to be about $5RC$  
\end{defi}


\begin{defi}[Magnetic fields]
\begin{equation*}
    \Phi=B\times A (\cos\theta)
\end{equation*}
\begin{defi}[Magnetic Field in the current]
Current in a wire produces a magnetic field, Identifies direction of B-field around either wire , Each wire is in the magnetic field of the other wire. A current-carrying wire in a magnetic field experiences a force, By Fleming's left hand rule.
\end{defi}
The current-carrying conductors: The direction of the F/B/I is given by fleming's left hand rule
\begin{equation*}
    F=B(\sin\theta) Il
\end{equation*}
\begin{defi}[Particle Movement in b-field]
 With a greater radius of curvature , use $r=\frac{p}{BQ}$ and left-hand Rule to deduce the direction and curvature. Rotation of Electromagnet: Rotor experiences a force and $F=BIl$ Due to the current in the rotor being in a magnetic field Or rotor becomes a magnet.
\end{defi}
If magnetic fields apply of charged(q) particle is produce circular motion Or provides a centripetal force Or causes spirals/arc.Direction of force indicates(sign of) charge.
Momentum/speed/mass found from radius by 
\begin{equation*}
     F=Bqv(\sin\theta)=\frac{mv^2}{r}
\end{equation*}
The radius of curve gets less because particle slows down.
\end{defi}

\begin{defi}[Electromagnetic Induction]
There is a magnetic field in iron core. This field/flux is changing due to the AC magnetic field, B-field passes through the rotor , Magnetic field lines are cut, The changing magnetic flux/field leads to an induced emf,Since the current is in a completed circuit.The amount of EMF induced can be found by Faraday's law and Lenz's Law\\
first you know:
\begin{defi}[Flux linkage]
\begin{equation*}
    N\Phi=NBA
\end{equation*}
\end{defi}
\end{defi}
\begin{law}[Faraday's law and Lenz's Law]
Faraday's law:states that the magnitude of the emf $\xi$ , is directly proportional to the rate of change of flux linkage
\begin{equation*}
    \xi\varpropto\frac{d(N\Phi)}{dt}
\end{equation*}
Lenz's Law:states that the induced emf must cause a current flow in such a direction as to oppose the change in flux linkage that produces it , otherwise energy would appear form nowhere. Put together:
\begin{equation*}
   \xi=-\frac{d(N\Phi)}{dt}
\end{equation*}
We always use Lenz's Law to calculate, the (-) is the direction of induced current and please look d as $\Delta$ such that $-\frac{N\Delta BA}{\Delta t}$
The direction of the induced emf can be found by right hand rule (motion/field/induced emf)
\end{law}





\begin{prop}[Way to increase power input]
Increase frequency of current and magnitude of current and Add more turns to either coil.
\end{prop}





\begin{defi}[The dynamo effect]
Generators, convert KE into electrical energy.\\
As the coil rotates , the magnetic flux linkage goes from zero(when the coil is parallel with the field) to a maximum(when the coil perpendicular to the field) An Alternating emf(A.C.) is generated.The voltage and current change direction with every half rotation.
\end{defi}

\begin{defi}[The transformer effect]
Transformer are devices that make use of electromagnetic to change the size of the voltage of an alternating current.

If the Alternating current flowing in a coil , then so does the magnetic field produced. If the changing magnetic flux links with another coil then there will be an emf induced in that coil.The coil in which current is changing is referred to as the primary. The other coil called the secondary.\\

Step-up transformers increase the voltage by having morn turns on the secondary coil than the primary.Step down transformer reduce the voltage by having fewer turns on the secondary coil.\\

The ensure maximum flux linkage between the two coil they are mounted on a SOFT iron core. This becomes magnetized when current flows in the primary, and flux is channelled to secondary.


\end{defi}

\begin{defi}[Eddy currents]
Any metal object placed in a region of changing magnetic field will have an emf induced in it. This emf causes current to circulate in the object. We call such currents eddy currents. The eddy currents in a transformer core dissipate energy $P=I^2R$ so can lead to inefficiency. In situations where relative movement is involved, the eddy currents cause braking.
\end{defi}

\subsection{Particle Physics}

\begin{defi}[Neutral Matters]
Does not leave a track and Charge conservation
\end{defi}

\begin{defi}[Electron guns]
In cathode ray tudes is used to produce a beam of fast velocity electrons(through large p.d.). A filament heats a metal cathode (by a current) which release electrons, resulting in thermionic emission().
\end{defi}

\begin{defi}[Electron]
Electrons do not just behave as particles- they also have wave properties.De Brogile's wave equation relates the $\lambda$, of a particle to its momentum $p$
\begin{equation*}
    \lambda=\frac{h}{p}
\end{equation*}
where h is the Planck constant\\

Fast electron or other particle may have a wavelength similar in size to nuclear matter, which means its structure can be investigated in scattering experiments. The higher the particle energy, the shorter the wavelength, hence the greater the detail that can be resolved. So high-energy particle beams are required for fine structure to be investigated.
\end{defi}

\begin{defi}[Cyclotron]
Electric field/p.d. accelerates particles by giving particles energy Constant time period.Polarity of dees switches every half cycle. Magnetic field /force at right angles to particles path.Maintains circular motion(whilst in dees).Radius of circle increase as particles get faster.
\end{defi}
\begin{defi}[Linear Accelerator]
Particles accelerate when in the gaps.p.d./polarity/supply reverses while particles are in the tube and switches at constant time interval and has a constant frequency.(Drift)tubes get longer so particles are in tubes for the same time.
\end{defi}




\begin{defi}[Availability of antimatter]
Availability of antimatter is poor and Difficulty of storing antimatter
\end{defi}
\begin{defi}][Alpha particles in Gold Nucleus/Alpha Scattering]
Observation:\\
Most alpha went straight through some deflected and very few came straight back.\\
Conclusion:\\
Atom mainly empty(space) and Charge and mass is concentrated in the center
\end{defi}

\begin{defi}[Particles-Antiparticles]
Charge conserved\\
Strangeness Conserved\\
Lepton Number Conserved\\
Momentum Conserved\\
Energy(mass) Conserved
\end{defi}





\section{Physics from creation}
\subsection{Thermal energy}

\begin{defi}[Internal energy]
is (sum of) molecular kinetic and potential energies.
the $E_k$ decide to change Temperature and $E_p$ decide to change state/shape.In an ideal gas the molecules have only kinetic energy. The temperature stop increasing at the point which is boiling point (gas $\iff$ liquid) or melting point (liquid $\iff$ solid)
\end{defi}

\begin{defi}[Specific heat capacity and latent heat]
\begin{equation*}
    E_k=mc\Delta\theta \mbox{~and~} E_p=ml
\end{equation*}
\end{defi}

\begin{defi}[Conservation of energy]
Heat loss $=$ Heat gain (include the latent heat(PE) and heat(KE))
(hot body loss energy to cold body)
\end{defi}

\begin{defi}[Gases]
$Ek\propto T$ above $0K$ , the air molecules are in continual.random motion, If the gas reached absolute zero then the K.E. of the molecules would be zero.Lower Temperature ehn lower collision rate and the change of momentum is decreased.
\end{defi}

\begin{defi}[Gas law]
\begin{equation*}
    PV=NkT
\end{equation*}
\end{defi}



\subsection{Nuclear decay}
\begin{defi}[Radioactive Isotope]
Unstable nucleus and emits $\alpha/\beta/\gamma$ radiation.
\end{defi}
\begin{defi}[Radioactive Decay]
We can't know when an individual nucleus will decay, and which nucleus will decay next
\end{defi}

\begin{defi}[Identification of particles]
\begin{enumerate}
    \item Record the background count
    \item Fix one source close to the GM-tube
    \item Introduce a sheet of paper or Aluminum between the source and the GM-tube.
    \item If the paper or Aluminum reduces the count rate to background levels then it is the objects emits aloha particles only.
    \item If the paper or Aluminum cause changes but the rate is still above the background rate , then the objects emits beta and gamma particles only.
    \item If the paper or Aluminum cause no change or little changes then it is the object emits gamma particles only.
\end{enumerate}
\end{defi}

\begin{defi}[Fusion]
Process:\\
Small nuclei fuse together to produce a larger nucleus, Mass of the fused nucleus $<$ total mass of initial nuclei.Energy is released as $\Delta E =mc^2$\\
Condition:\\
A very high temperature and pressure.To overcome the electrostatic repulsion between nuclei.To maintain a high/sufficient collision rate.\\
Difficulty:\\
Contact with container causes temperature to fall. Very strong magnetic field required.\\

The B.E. per nucleon for $He^4$is greater/higher/larger (than other small nuclei) which means $He^4$ nucleus is the most stable (of he small nulei)
\end{defi}
\begin{defi}[Fission]
Some heavy nuclei can undergo induced fission Or massive nuclei can be made to split into smaller nuclei. Massive nuclei have less B.E. per nucleon than the less massive nuclei produced in the fission . Hence energy is released in the fission. Mass of reactant $>$ Mass of products. Energy is released according to $E=mc^2$
\end{defi}

\subsection{Oscillations}
\begin{defi}[Simple Harmonic Motion]
Acceleration is directly proportional to displacement form equilibrium position.Acceleration is in the opposite direction to displacement.
\end{defi}
\begin{defi}[Resonance]
The second object is forced to vibrate at the frequency of the first object.Because they have similar natural frequency.Driver/forcing frequency Matched natural frequency.Maximum Energy is transferred from the first object to the second object.Larger rate of transfer of energy means that the vibration persists for a shorter time.
\end{defi}

\begin{defi}[Damping]
Energy of the system is dissipated or energy is removed from the system, Hence the amplitude reduces.
\end{defi}

\subsection{Astrophysics and cosmology}
Methods determine the distance of the stars
\begin{defi}[Parallax method]
Angles are measured using the fixed background of more distant stars.\\
Find angular displacement of the star as Earth moves around the Sun over a 6 month period. The diameter/ radius of the Earth's orbit about the Sun must be known and trigonometry is used.
\end{defi}

\begin{defi}[Standard candle]
Flux/brightness/intensity of standard candle is measured.Luminosity of standard candle is known.Inverse square law is used.
\end{defi}

\begin{defi}[Dark Matter]
It cannot be detected via the em-interaction.But it has mass and exerts a gravitational force
\end{defi}

\begin{defi}[Universe expanding]
The density of the universe may be greater than the critical density. Hence the universe is more likely to reach a maximum size and more likely to be closed. If the density is approximately equal to the critical density, The expansion will slow down and the size will keep constant . If the density is far less than the critical density, The universe will keep expanding forever
\end{defi}

\begin{defi}[The Death of the Sun]
The Sun is fusing/burning hydrogen(into helium in its core) When (hydrogen) fusion/burning ceases the core of the Sun cools.The core collapses/contracts(under gravitational forces) The sun expands and becomes a red giant. The core becomes hot enough for helium fusion/burning to begin (in the core) Helium begins to run out and the core collapses again (under gravitational forces) Outer layers of Sun are ejected into space . The temperature doesn't rise enough for further fusion to begin.
\end{defi}

\begin{defi}[White Dwarfs]
They are the core remnant of a red giant star. There is no fusion in the white dwarf . They have a relatively small surface area so they are not very luminous by black body equation.They are very hot and appear white because they emits all visible wavelengths.
\end{defi}

\begin{defi}[Originate of the Microwaves]
Originates form the Big Bang, Microwave radiation comes form the universe itself, Microwave wavelength shows the temperature of universe , which indicates a temperature of space of about 3K by Wien's Law. Temperature decrease as the universe expands. Wavelength has been increased.
\end{defi}

\begin{defi}[Start of the Universe]
The universe started from a mall initial point. Idea that universe has a finite age. Idea that (observable universe is finite because)
we can only see.As far as (speed of light)$\times$(age of universe)
Geostationary Satellites. T smaller than 24 hrs , Increase radius, and vice versa. Because $GM=rv^2$ and $GM$ is constant
\end{defi}

\section{Skills}
\begin{prop}[Unit]

\end{prop}
\begin{prop}[Source of Error]
Parallax\\
(Human) reaction time\\
Knowing exact point it passes markers\\
Zero error in stopwatch(And measuring tools' accuracy)
\end{prop}


\begin{defi}[Independent Variable and Dependent Variables]
\end{defi}

\end{document}


\end{document}