\documentclass[a4paper]{article}

\def\npart{III}
\def\nterm {Lent}
\def\nyear {2017-2018}
\def\nlecturer {Brian}
\def\ncourse {Mechanics}

\input{header}

\begin{document}

\maketitle


\tableofcontents

\section{Kinematics of a particles moving in a straight line}

\section{Dynamics of a particle moving in a straight line}

\section{Statics of a particle}

\section{Moments}

\section{Vectors}

\section{Kinematics of a particle moving in a straight line or plane}

\section{Centres of mass}

\section{Work , energy and power}

\section{Collisions}

\section{Statics of rigid bodies 1}

\section{Further kinematics}
\subsection{Forces which vary with speed}
\begin{prop}
	\[
		\mathbf{a}=\mathbf{v}\frac{\d \mathbf{v}}{\d \mathbf{x}}
	\]
	\begin{proof}
		\[
			\mathbf{a}=\frac{\d \mathbf{x}}{\d \mathbf{t}}\times \frac{\d \mathbf{v}}{\d \mathbf{x}}=\mathbf{v}\frac{\d \mathbf{v}}{\d \mathbf{x}}
		\]
	\end{proof}
\end{prop}
\section{Elastic strings and springs}
\subsection{Hooke's Law}
\begin{law}[Hooke's Law]
	There are two cases for using Hooke's Law
	\begin{enumerate}
		\item Elastic strings: The tension $T$ in an elastic string is
		      \[
			      T=\frac{\lambda x}{l}
		      \]
		      where \\
		      $l$ is the natural (unstretched) length of the string,\\
		      $x$ is the extension and\\
		      $\lambda$ is the modulus of elasticity
		      \begin{center}
			      \includegraphics[scale=0.5]{img_M/12_intro1}
		      \end{center}
		\item Elastic springs: The tension, or thrust, $T$ is an elastic spring is
		      \[
			      T=\frac{\lambda x}{l}
		      \]
		      where \\
		      $l$ is the natural (unstretched) length of the string,\\
		      $x$ is the extension or compression and\\
		      $\lambda$ is the modulus of elasticity
		      \begin{center}
			      \includegraphics[scale=0.5]{img_M/12_intro2}
		      \end{center}
	\end{enumerate}
\end{law}

\subsection{Energy stored in an elastic string or spring}
Like kinematics,If there is force $F$ and displacement traveled $\delta s$, the Work done is $\delta W=F\delta s$. Similarly, If the tension force is $T$ and string/spring extended/stretched, then
\[
	\delta W \approx T \delta x
\]
Total work done in exrending from $x=0$ to $x=X$ is approximately
\[
	\sum^{X}_0 T \delta x
\]

and , as $\delta x \rightarrow 0 $, the total work done:

\[
	W=\int^X_0 T \d x= \int^X_0 \frac{\lambda x}{l}\d x=\frac{\lambda x^2}{2l}
\]
The expression of Total work done is also called the Elastic Potential Energy

\section{Further dynamics}
\subsection{Impulse of a variable force}
\[
	\delta I \approx F(t) \delta t
\]
The total impulse from time $t_1$ to $t_2$ is 
\[
    I \approx \sum_{t_1}^{t_2} F(t) \delta t
\]
and as $\delta t \rightarrow 0$, the total impulse is 
\[
    I=\int^{t_2}_{t_1} F(t) \d t
\]
Also, as $F(t)=ma=m\frac{\d v}{\d t}$
\begin{align*}
    \int_{t_1}^{t_2} F(t) \d t &=\int^{V}_{U} m \d v =mV-mU
\end{align*}
\subsection{Work done by a variable force}
\[
    \delta W \approx G(x)\delta x
\]
and the total work done in moving from a displacement $x_1$ to $x_2$ is
\[
    W\approx\sum_{x_1}^{x_2} G(x)\delta x
    \]
and as $\delta x \rightarrow 0$ , the total work done is
\[
    W=\int_{x_1}^{x_2} G(x) \d x
\]
Also $G(x)=ma=m\frac{\d v}{\d x}=m\frac{\d x}{\d t}\times \frac{\d v}{\d x}=mv\frac{\d v}{\d x}$
\[
    \int_{x_1}^{x_2} G(x) \d x =\int_U^V mv \d v =\frac{1}{2}mV^2-\frac{1}{2}mU^2
\]

\subsection{Newton's Law of Gravitation}
\begin{law}
    The force of attraction between two bodies of masses  
$M_1$
  and  
$M_2$
  is directly proportional to the product of their masses and inversely proportional to the square of the distance, 
$d$ , between them:
\[
    F=\frac{GM_1M_2}{d^2}
    \]
where $G$ is a constant known as the constant of Gravitation
\end{law}
\subsection{Finding $k$ in $F=\frac{k}{x^2}$}
\[
    F=ma=\frac{k}{d^2}
\]

\subsection{Simple harmonic motion S.H.M.}

\section{Motion in a circle}
\subsection{Angular velocity}

\subsection{Acceleration}

Types of problems

\subsection{Motion in a vertical circle}

Types of problems
\section{Statics of rigid bodies 2}
\subsection{Centre of mass}
\section{Relative motion}

\section{Elastic collisions in two dimensions}

\section{Resisted motion of a particle moving in a straight line}

\section{Damped and forced harmonic motion}

\section{Stability}

\section{Applications of vectors in mechanics}

\section{Variable mass}

\section{Moments of inertia of a rigid body}

\section{Rotation of a rigid body about a fixed smooth axis}




\printindex



\end{document}}