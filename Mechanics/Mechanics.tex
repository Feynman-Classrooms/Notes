\documentclass[a4paper]{article}

\def\npart{III}
\def\nterm {Lent}
\def\nyear {2017-2018}
\def\nlecturer {Brian}
\def\ncourse {Mechanics}

\input{header}

\begin{document}

\maketitle


\tableofcontents

\section{Kinematics of a particles moving in a straight line}

\section{Dynamics of a particle moving in a straight line}

\section{Statics of a particle}

\section{Moments}

\section{Vectors}

\section{Kinematics of a particle moving in a straight line or plane}

\section{Centres of mass}

\section{Work , energy and power}

\section{Collisions}

\section{Statics of rigid bodies 1}

\section{Further kinematics}
\subsection{Forces which vary with speed}
\begin{prop}
	\[
		\mathbf{a}=\mathbf{v}\frac{\d \mathbf{v}}{\d \mathbf{x}}
	\]
	\begin{proof}
		\[
			\mathbf{a}=\frac{\d \mathbf{x}}{\d \mathbf{t}}\times \frac{\d \mathbf{v}}{\d \mathbf{x}}=\mathbf{v}\frac{\d \mathbf{v}}{\d \mathbf{x}}
		\]
	\end{proof}
\end{prop}
\section{Elastic strings and springs}
\subsection{Hooke's Law}
\begin{law}[Hooke's Law]
	There are two cases for using Hooke's Law
	\begin{enumerate}
		\item Elastic strings: The tension $T$ in an elastic string is
		      \[
			      T=\frac{\lambda x}{l}
		      \]
		      where \\
		      $l$ is the natural (unstretched) length of the string,\\
		      $x$ is the extension and\\
		      $\lambda$ is the modulus of elasticity
		      \begin{center}
			      \includegraphics[scale=0.5]{img_M/12_intro1}
		      \end{center}
		\item Elastic springs: The tension, or thrust, $T$ is an elastic spring is
		      \[
			      T=\frac{\lambda x}{l}
		      \]
		      where \\
		      $l$ is the natural (unstretched) length of the string,\\
		      $x$ is the extension or compression and\\
		      $\lambda$ is the modulus of elasticity
		      \begin{center}
			      \includegraphics[scale=0.5]{img_M/12_intro2}
		      \end{center}
	\end{enumerate}
\end{law}

\subsection{Energy stored in an elastic string or spring}
Like kinematics,If there is force $F$ and displacement traveled $\delta s$, the Work done is $\delta W=F\delta s$. Similarly, If the tension force is $T$ and string/spring extended/stretched, then
\[
	\delta W \approx T \delta x
\]
Total work done in exrending from $x=0$ to $x=X$ is approximately
\[
	\sum^{X}_0 T \delta x
\]

and , as $\delta x \rightarrow 0 $, the total work done:

\[
	W=\int^X_0 T \d x= \int^X_0 \frac{\lambda x}{l}\d x=\frac{\lambda x^2}{2l}
\]
The expression of Total work done is also called the Elastic Potential Energy

\section{Further dynamics}
\subsection{Impulse of a variable force}
\[
	\delta I \approx F(t) \delta t
\]
The total impulse from time $t_1$ to $t_2$ is
\[
	I \approx \sum_{t_1}^{t_2} F(t) \delta t
\]
and as $\delta t \rightarrow 0$, the total impulse is
\[
	I=\int^{t_2}_{t_1} F(t) \d t
\]
Also, as $F(t)=ma=m\frac{\d v}{\d t}$
\begin{align*}
	\int_{t_1}^{t_2} F(t) \d t & =\int^{V}_{U} m \d v =mV-mU
\end{align*}
\subsection{Work done by a variable force}
\[
	\delta W \approx G(x)\delta x
\]
and the total work done in moving from a displacement $x_1$ to $x_2$ is
\[
	W\approx\sum_{x_1}^{x_2} G(x)\delta x
\]
and as $\delta x \rightarrow 0$ , the total work done is
\[
	W=\int_{x_1}^{x_2} G(x) \d x
\]
Also $G(x)=ma=m\frac{\d v}{\d x}=m\frac{\d x}{\d t}\times \frac{\d v}{\d x}=mv\frac{\d v}{\d x}$
\[
	\int_{x_1}^{x_2} G(x) \d x =\int_U^V mv \d v =\frac{1}{2}mV^2-\frac{1}{2}mU^2
\]

\subsection{Newton's Law of Gravitation}
\begin{law}
	The force of attraction between two bodies of masses
	$M_1$
	and
	$M_2$
	is directly proportional to the product of their masses and inversely proportional to the square of the distance,
	$d$ , between them:
	\[
		F=\frac{GM_1M_2}{d^2}
	\]
	where $G$ is a constant known as the constant of Gravitation
\end{law}
\subsection{Finding $k$ in $F=\frac{k}{x^2}$}
\[
	F=ma=\frac{k}{d^2}
\]

\subsection{Simple harmonic motion S.H.M.}

\begin{defi}[S.H.M. equation]
	If a particle
	,
	$P$
	, moves in a straight line so that
	its acceleration is proportional to its distance
	from a fixed point
	$O$
	, and directed towards
	$O$
	,
	then
	\[
		\ddot{x}=-\omega^2x
	\]
	and the particle will oscillate between two points,
	$A$
	and
	$B$
	, with simple harmonic motion.\\

	The amplitude of the oscillation is
	$OA = OB= a$.\\

	Notice that $\ddot{x}$
	is marked
	in the direction of
	$x$
	increasing
	$n$ the diagram
	, and, since
	$\omega^2$
	is
	positive,
	$\ddot{x}$
	is negative
	, so the acceleration acts towards
	$O$.
	\begin{center}
		\includegraphics[scale=0.5]{img_M/12_intro4}
	\end{center}
\end{defi}

\begin{prop}[Solving equation]
	A.E. is
	\[
		m^2=-\omega^2 \rightarrow m=i\omega
	\]
	G.S. is
	\[
		x=\lambda \sin \omega t+\mu \cos \omega t
	\]
	If $x$ starts from $O$, $x=O$ when $t=0$, \\
	then
	\[
		x=a \sin\omega t
	\]
	If $x$ starts from $B$, $x=a$ when $t=0$, \\
	then
	\[
		x=a\cos\omega t
	\]
\end{prop}
\begin{defi}[Period and amplitude]
	From the equations $x=a \sin\omega t$ and $x=a\cos\omega t$\\
	we can see that the period, the time for one complete oscillation, is
	\[
		T=\frac{2\pi}{\omega}
	\]
	The period is the time taken to go from $O\rightarrow B\rightarrow A\rightarrow O$, or from $B\rightarrow A \rightarrow B$ and that the
	amplitude, maximum distance from the central point, is $a$.
\end{defi}

\begin{prop}[Alternative equation of S.H.M.]
	\[
		v^2=\omega^2(a^2-x^2)
	\]
	\begin{proof}
		Consider the basic S.H.M. equation $\ddot{x}=-\omega^2x$ and $\ddot{x}=v \frac{\d v}{\d x}$
		\begin{align*}
			v \frac{\d v}{\d x} & =-\omega^2x                           \\
			\int v \d v         & =\int -\omega^2 x \d x                \\
			\frac{1}{2}v^2      & =-\frac{1}{2}\omega^2x^2+\frac{1}{2}c
		\end{align*}
		But $v=)$ when $x$ at its maxumum, $x= a \rightarrow c=a^2\omega^2$
		\begin{align*}
			\frac{1}{2}v^2 & =-\frac{1}{2}\omega^2x^2+\frac{1}{2}a^2\omega^2 \\
			v^2            & =\omega^2(a^2-x^2)
		\end{align*}
	\end{proof}
\end{prop}
Horizontal
\begin{eg}
\end{eg}
Vertical (relate to mg)
\begin{eg}
\end{eg}
\section{Motion in a circle}
\subsection{Angular velocity}
A particle moves in a circle of radius $r$ with constant speed, $v$.\\

Suppose that in a small time $\delta t$ the particle moves through a small angle $\delta\theta$, then the distance moved will be $\delta s = r\delta \theta$
\\
and its speed $v=\frac{\delta s}{\delta t}=r\frac{\delta \theta}{\delta t}$\\

and , as $\delta t \rightarrow 0$, $v=r\frac{\d \theta}{\d t}=r\theta$

\[
	\frac{\d\theta}{\d t}=\theta
\]
is the angular velocity, usually written as the Greek letter omega, $\omega$, and so, for a particle moving in a circle with radius $r$, its speed is
\[
	v=r\omega
\]

\begin{center}
	\includegraphics[scale=0.5]{img_M/13_intro1}
\end{center}

\subsection{Acceleration}
A particle moves in a circle of radius $r$ with constant speed , $v$.\\

Suppose that in a small time $\delta t$ the particle moves through a small angle $\delta\theta$, and that its velocity changes from $v_1$ to $v_2$, \\

then its change in velocity is $\delta v =v_2-v_1$, which is shown in the second diagram. \\

The lengths of both $v_1$ and $v_2$ are $v$, and the angle between $v_1$ and $v_2$ is $\delta\theta$.
\begin{align*}
	\delta v                  & = 2\times v \sin \frac{\delta\theta}{2}\approx 2v\times\frac{\delta\theta}{2}=v\delta\theta
	\frac{\delta v}{\delta t} & \approx v\frac{\delta\theta}{\delta t}
\end{align*}
as $\delta t \rightarrow 0 $, acceleration:
\[
	a=\frac{\d v}{\d t}=v\frac{\d \theta}{\d t}=v\theta
\]
But
\[
	\theta=\omega=\frac{v}{r} \rightarrow a = \frac{v^2}{r}=r\omega^2
\]
Notice that as $\delta\theta \rightarrow 0$, the direction of $\delta v$ becomes perpendicular to both $v_1$ and $v_2$, and so is directed towards the centre of the circle.\\

The acceleration of a particle moving in a circle with speed $v$ is $a=r\omega^2=\frac{v^2}{r}$, and is directed towards the centre of the circle.\\

Alternative proof
\begin{proof}
	If a particle moves , with constant speed, in a circle of radius $r$ and centre $O$, then its position vector can be written:
	\[
		\mathbf{r}=r\begin{pmatrix} \cos\theta \\ \sin\theta \end{pmatrix} \rightarrow \mathbf{\dot{r}}=r\begin{pmatrix}-\sin\theta \dot{\theta}\\ \cos\theta\dot{\theta}\end{pmatrix}
	\]
	Particle moves with constant speed $\rightarrow \dot{\theta}=\omega$ is constant
	\[
		\mathbf{\dot{r}}=r\begin{pmatrix}-\sin\theta \\ \cos\theta \end{pmatrix} \rightarrow v=r\omega
	\]
	\[
		\mathbf{\ddot{r}}=r\omega\begin{pmatrix}-\cos\theta \dot{\theta}\\ -\sin\theta\dot{\theta}\end{pmatrix}=-\omega^2r\begin{pmatrix} \cos\theta \\ \sin\theta \end{pmatrix}=-\omega^2\mathbf{r}
	\]
	acceleration is

	\[
		r\omega^2 \mbox{~or~} \frac{v^2}{r}
	\]
	directed towards $O$.
\end{proof}
\begin{center}
	\includegraphics[scale=0.5]{img_M/13_intro2}
\end{center}

Types of problems:
\begin{enumerate}
	\item Horizontal
    \item Conical pendulum
    \begin{center}
        \includegraphics[scale=0.5]{img_M/13_eg1}
    \end{center}
	\item Banking
    \begin{center}
        \includegraphics[scale=0.5]{img_M/13_eg2}
    \end{center}
	\item Inside an inverted vertical cone
    \begin{center}
        \includegraphics[scale=0.5]{img_M/13_eg3}
    \end{center}
\end{enumerate}

\subsection{Motion in a vertical circle}

\begin{prop}
	\[
		a=\frac{v^2}{r}
	\]
	\begin{proof}
		If a particle moves in a circle of radius $r$ and centre $O$, then its position vector can be written:
		\[
			\mathbf{r}=r\begin{pmatrix}\cos\theta\\ \sin\theta \end{pmatrix}
        \]
        \begin{align*}
        \mathbf{\dot{r}}&=r\begin{pmatrix}-\sin\theta \dot{\theta}\\ \cos\theta\dot{\theta}\end{pmatrix}=r\dot{\theta}\begin{pmatrix}-\sin\theta\\ \cos\theta \end{pmatrix}\\
        \mathbf{\ddot{r}}=r\begin{pmatrix}-\cos\theta\dot{\theta}^2-\sin\theta\ddot{\theta}\\-\sin\theta\dot{\theta}^2+\cos\theta\ddot{\theta}\end{pmatrix}=-r\dot{\theta}^2\begin{pmatrix}\cos\theta\\\sin\theta\end{pmatrix}+r\ddot{\theta}\begin{pmatrix}-\sin\theta\\\cos\theta\end{pmatrix}
        \end{align*}

        From this we can see that the speed is $v=r\dot{\theta}=r\omega$,\\
        and is perpendicular to the radius since $\mathbf{r\cdot\dot{r}}=0$\\
        
        We can also see that the acceleration has two components
        \[
            r\dot{\theta}^2=r\omega^2=\frac{v^2}{r}
        \]
        towards the centre opposite direction to $\mathbf{r}$\\
        
        and $r\ddot{\theta}$ perpendicular to the radius which is what we should expect since $v=r\dot{\theta}$ and $r$ is constant.\\
        
        In practice we shall onlu use 
        \[
            a=r\omega^2=\frac{v^2}{r}
        \]
        directed towards the centre of the circle
	\end{proof}
\end{prop}

Types of problems
\begin{enumerate}
    \item A particle attached to an inextensible string
    \begin{center}
        \includegraphics[scale=0.5]{img_M/13_eg4_1}
    \end{center}
    \begin{center}
        \includegraphics[scale=0.5]{img_M/13_eg4_2}
    \end{center}
	\item A particle moving on the indside of a smooth, hollow sphere
    \begin{center}
        \includegraphics[scale=0.5]{img_M/13_eg5}
    \end{center}
	\item A particle attached to a rod
    \begin{center}
        \includegraphics[scale=0.5]{img_M/13_eg6}
    \end{center}
	\item A particle moving on the outside of a smooth sphere
    \begin{center}
        \includegraphics[scale=0.5]{img_M/13_eg7}
    \end{center}
\end{enumerate}
\section{Statics of rigid bodies 2}
\subsection{Centre of mass}
When finding a centre of mass \\

Centres of mass depend on the formula :
\[
    M\bar{x}=\sum m_ix_i
\]
or Similar, Remember that 
\[
    \lim_{\delta x\rightarrow 0}\sum f(x_i)\delta x=\int f(x) \d x
\]

\subsection{Centre of mass of geometric shapes}
\subsubsection{Sector}
In this case we can find a nice method, using the result for the centre of mass of a triangle.
\\
We take a sector of angle $2\alpha$ and divide it into many smaller sectors.\\

Mass of whole sector 
\[
    M=\frac{1}{2}r^2\times 2\alpha\times\rho=r^2\alpha\rho
\]
Consuder each small sector as approximately a triangle, with centre of mass, $G_1$, $2/3$ along the median from $O$\\

Working in polar coordinates for one small sector,
\[
    m_i=\frac{1}{2}r^2\rho\delta\theta
\]
\[
    OP=r\rightarrow OG_1 \cong \frac{2}{3}r \rightarrow x_i \cong \frac{2}{3}r\cos\theta 
\]
\begin{align*}
    \lim_{\delta\theta\rightarrow 0}\sum_{\theta=-\alpha}^{\alpha}m_ix_i&=\int_{-\alpha}^{\alpha}\frac{1}{2}r^2\rho\times \frac{2}{3}r\cos\theta\d\theta\\ 
    &=\frac{2}{3}r^3\rho\sin\alpha\\
    \bar{x}&=\frac{\sum m_ix_i}{M}=\frac{\frac{2}{3}r^3\rho\sin\alpha}{r^2\alpha\rho}=\frac{2r\sin\alpha}{3\alpha}
\end{align*}
By symmetry, $\bar{y}=0$\\
centre of mass is at $(\frac{2r\sin\alpha}{3\alpha},0)$
\begin{center}
    \includegraphics[scale=0.5]{img_M/14_intro1}
\end{center}
\subsubsection{Circular arc}
For a circular arc of radius $r$ which subtends an angle of $2\alpha$ at the centre. \\

The length of the arc is $r\times 2\alpha$ \\
The mass of the arc is $M=2\alpha r \rho$ \\

First divide the arc into several small pieces, each subtending an angle of $\delta\theta$ at the centre \\

The length of each piece is $r\delta\theta\rightarrow m_i=r\rho\delta\theta$\\

We now think of each small arc as a point mass at the centre of the arc, with $x$-corrdinate $x_i=r\cos\theta$\\

\begin{align*}
    \lim_{\delta\theta\rightarrow0}\sum_{\theta=-\alpha}^{\alpha}m_ix_i&=\int_{-\alpha}^{\alpha}r\rho\times r \cos\theta\d\theta \\
    &= 2r^2\rho\sin\alpha \\
    \bar{x}&=\frac{\sum m_ix_i}{M}=\frac{2r^2\rho\sin\alpha}{2r\alpha\rho}=\frac{r\sin\alpha}{\alpha}
\end{align*}
By symmetry, $\bar{y}=0$\\
centre of mass is at $(\frac{r\sin\alpha}{\alpha},0)$
\begin{center}
    \includegraphics[scale=0.5]{img_M/14_intro2}
\end{center}

\subsubsection{Others}
\begin{center}
    \includegraphics[scale=0.5]{img_M/14_others}
\end{center}

\subsubsection{Solid of revolution}
\begin{center}
    \includegraphics[scale=0.5]{img_M/14_eg1}
\end{center}
\subsubsection{Hemispherical shell}
\begin{center}
    \includegraphics[scale=0.5]{img_M/14_eg2}
\end{center}
\begin{center}
    \includegraphics[scale=0.5]{img_M/14_eg2_2}
\end{center}
\begin{center}
    \includegraphics[scale=0.5]{img_M/14_eg2_3}
\end{center}
\begin{center}
    \includegraphics[scale=0.5]{img_M/14_eg2_4}
\end{center}
\subsubsection{conical}
\begin{center}
    \includegraphics[scale=0.5]{img_M/14_intro3}
\end{center}
\subsubsection{Square based pyramid}
\begin{center}
    \includegraphics[scale=0.5]{img_M/14_intro4}
\end{center}
\subsubsection{The standard results}
\begin{center}
    \includegraphics[scale=0.5]{img_M/14_intro5}
\end{center}
\begin{center}
    \includegraphics[scale=0.5]{img_M/14_intro6}
\end{center}

\subsubsection{Tilting and hanging freely}

\begin{center}
    \includegraphics[scale=0.5]{img_M/14_intro7}
\end{center}
\begin{center}
    \includegraphics[scale=0.5]{img_M/14_eg3}
\end{center}

\subsubsection{Hemisphere in equilibrium on a slope}
\begin{center}
    \includegraphics[scale=0.5]{img_M/14_eg4}
\end{center}

\section{Relative motion}
\begin{defi}[relative displacement and velocity]
    which is direction vector of two position vectors.\\
    Velocity is diifferentiating the direction vector
\end{defi}

There are four types of questions
\begin{enumerate}
\item Collisions
\item Closest distance
\item Best course
\item Change in apparent direction of wind or current
\end{enumerate}

\begin{center}
    \includegraphics[scale=0.5]{img_M/15_eg1}
\end{center}
\begin{center}
    \includegraphics[scale=0.5]{img_M/15_eg2}
\end{center}
\begin{center}
    \includegraphics[scale=0.5]{img_M/15_eg3}
\end{center}
\begin{center}
    \includegraphics[scale=0.5]{img_M/15_eg4}
\end{center}

\section{Elastic collisions in two dimensions}
\subsection{Impulse}
\subsection{Momemtum}

\section{Resisted motion of a particle moving in a straight line}

\section{Damped and forced harmonic motion}
\subsection{Damped (Homogenuous)}
\subsection{Forced (In homogenuous)}

\section{Stability}
\begin{center}
    \includegraphics[scale=0.5]{img_M/20_eg1}
\end{center}
\begin{center}
    \includegraphics[scale=0.5]{img_M/20_eg2}
\end{center}
\begin{center}
    \includegraphics[scale=0.5]{img_M/20_eg3}
\end{center}


\section{Applications of vectors in mechanics}
\begin{center}
    \includegraphics[scale=0.5]{img_M/21_eg1}
\end{center}
\begin{center}
    \includegraphics[scale=0.5]{img_M/21_eg2}
\end{center}

\section{Variable mass}
\subsection{Examples}
\begin{center}
    \includegraphics[scale=0.5]{img_M/22_eg1}
\end{center}
\begin{center}
    \includegraphics[scale=0.5]{img_M/22_eg2}
\end{center}
\begin{center}
    \includegraphics[scale=0.5]{img_M/22_eg3}
\end{center}

\section{Moments of inertia of a rigid body}
\begin{center}
    \includegraphics[scale=0.5]{img_M/23_eg1}
\end{center}
\begin{center}
    \includegraphics[scale=0.5]{img_M/23_eg2}
\end{center}
\begin{center}
    \includegraphics[scale=0.5]{img_M/23_eg3}
\end{center}
\begin{center}
    \includegraphics[scale=0.5]{img_M/23_eg4}
\end{center}
\begin{center}
    \includegraphics[scale=0.5]{img_M/23_eg5}
\end{center}
\begin{center}
    \includegraphics[scale=0.5]{img_M/23_eg6}
\end{center}
\begin{center}
    \includegraphics[scale=0.5]{img_M/23_eg7}
\end{center}


\section{Rotation of a rigid body about a fixed smooth axis}

\begin{center}
    \includegraphics[scale=0.5]{img_M/24_eg1}
\end{center}
\begin{center}
    \includegraphics[scale=0.5]{img_M/24_eg2}
\end{center}
\begin{center}
    \includegraphics[scale=0.5]{img_M/24_eg3}
\end{center}
\begin{center}
    \includegraphics[scale=0.5]{img_M/24_eg4}
\end{center}
\begin{center}
    \includegraphics[scale=0.5]{img_M/24_eg5}
\end{center}
\begin{center}
    \includegraphics[scale=0.5]{img_M/24_eg6}
\end{center}



\printindex



\end{document}}