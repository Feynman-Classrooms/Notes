\documentclass[a4paper]{article}

\def\npart{III}
\def\nterm {Lent}
\def\nyear {2017-2018}
\def\nlecturer {Brian}
\def\ncourse {Statistics}

\input{header}

\begin{document}

\maketitle


\tableofcontents

\section{Representation and summary of data - location}
\subsection{Basic Concepts of Variable}
\begin{defi}[Quantitative variables and Qualitative variables]
	Quantitative variable associated with numerical observation. Qualitative variables associated with non-numerical observations.
\end{defi}

\begin{defi}[Continuous variable and discrete variable]
	Continuous variable can take ant value in given range. Discrete can take only specific values in a given range.
\end{defi}


\subsection{Grouped data}
\begin{defi}[Grouped data]
	The groups are more commonly known as classes.
	\begin{itemize}
		\item class boundaries.
		\item mid-point of a class.
		\item class width.
	\end{itemize}
\end{defi}

\begin{eg}
	Example 5-6
\end{eg}

\begin{defi}[Frequency and cumulative frequency]
	Number of anything; example is how many sheeps.
	It is sometimes helpful to add a column to the table showing the running total of the frequencies. This is called the cumulative frequency
\end{defi}


\begin{defi}[Ungrouped data]
	Show all data
\end{defi}

\subsection{Mean , mode and median}
\begin{defi}[Mode]
	The mode is the value that occurs most often
\end{defi}

\begin{defi}[Median]
	n/2 term or 1 term above
\end{defi}

\begin{defi}[Mean]
	\[
		\bar{x}=\frac{\sum_i^n x_i}{n}
	\]
\end{defi}

\subsection{Linear interpolation}
\begin{eg}
	Example 14-15
\end{eg}

\subsection{Coding}
\begin{eg}
	pick 1 example
\end{eg}


\section{Representation and summary of data - measures of dispersion}
\subsection{Range and interquartile range}
The list of formula:
\begin{itemize}
	\item Range = Upper value $-$ Lowest value
\end{itemize}
\begin{eg}
	example 3
\end{eg}

\subsection{Percentiles split the data into 100 parts}
\begin{eg}
	example 4
\end{eg}

\subsection{Range and Interquartile range}

\begin{eg}[Linear Interpolation]

\end{eg}

\subsection{Variance and standard deviation}
\begin{defi}[Variance]
	Let $f$ stand for the frequency, then $n=\sum f$ and
	\[
		\text{Variance}=\frac{\sum f(x-\bar{x})^2}{\sum f} \mbox{~or~} \frac{\sum fx^2}{\sum f}-(\frac{\sum fx^2}{\sum f})
	\]
\end{defi}

\subsection{Variance and standard deviation for grouped data}
\begin{defi}

\end{defi}

\begin{eg}
	example 7-8
\end{eg}

\subsection{Coding}

\begin{eg}
	example 9-11
\end{eg}

\section{Representation of data}

\subsection{Stem and Leaf diagrams}

\subsection{Outlier}
\begin{defi}
	An outlier is an extreme value that lies outside the overall pattern of the data.
\end{defi}
An outlier is any value, which is
\[
	\text{greater than the upper quartile} + 1.5 \times \text{interquartile range}
\]
OR
\[
	\text{less than the lower quartile} + 1.5 \times \text{interquartile range}
\]

\subsection{Box plot}
\begin{center}
	\includegraphics[scale=0.5]{img_S/4_intro}
\end{center}

\subsection{Histogram}

\begin{defi}[Frequency density]
	\[
		\text{frequency density}=\frac{\text{frequency}}{\text{class width}}
	\]
\end{defi}
\begin{eg}
	7
\end{eg}

\subsection{Skewness (Shape)}
A distribution can be symmetrical , have positive skew or have negative skew\\


symmetrical $Q_2-Q_1=Q_3-Q_2$ or mode$=$median$=$mean
\begin{center}
	\includegraphics[scale=0.5]{img_S/3_5_intro1}
\end{center}
positive :$Q_2-Q_1<Q_3-Q_2$ or mode$<$median$<$mean
\begin{center}
	\includegraphics[scale=0.5]{img_S/3_5_intro2}
\end{center}
negative :$Q_2-Q_1>Q_3-Q_2$ or mode$>$median$>$mean
\begin{center}
	\includegraphics[scale=0.5]{img_S/3_5_intro3}
\end{center}

Or you can calculate:
\[
	\frac{3(\text{mean}-\text{median})}{\text{SD}}
\]


\subsection{What!?}
\begin{eg}
	example 10-12
\end{eg}
\section{Probability}
\subsection{Classical Probability}
\subsection{Venn diagram and their rules}
\begin{defi}[Complementary Probability]
\end{defi}
\subsection{Conditional Probabilites}
\subsubsection{Vann diagram}
\subsubsection{Tree diagram}
\subsection{Special Events of Probabilites}

\begin{defi}[Mutually exclusive]
	When events have no outcomes in common, they are mutually exclusive.
	\begin{center}
		\includegraphics[scale=0.5]{img_S/4_6_intro}
	\end{center}

\end{defi}
There is no intersection of A and B, so $P(A\cap B)=0$\\

We can use $P(A\cup B)=P(A)+P(B)-P(A\cap B)$

result is
\[
	P(A\cup B)=P(A)+P(B)
\]

\begin{defi}[Independent events]
	When one event has no effect on another, they are independent so $P(A|B)=P(A)$
\end{defi}

by $\frac{P(A\cap B)}{P(B)}=P(A)$ we have:

\[
	P(A\cap B)=P(B)\times P(A)
\]



\section{Correlation}
\subsection{Correlation}
\begin{center}
	\includegraphics[scale=0.5]{img_S/5_1_intro}
\end{center}
\subsection{Bivariate data}
Recall this formula :
\[
	\text{Variance}=\frac{\sum (x-\bar{x})^2}{n}
\]
In correlation we write:
\[
	S_{xx}=\sum (x-\bar{x})^2
\]

\[
	S_{yy}=\sum (y-\bar{y})^2
\]

so
\[
	\text{Variance}=\frac{S_{xx}}{n}
\]

\begin{defi}[Co-Variance]
	\[
		S_{xy}=\frac{\sum(x-\bar{x})(x-\bar{y})}{n}
	\]
\end{defi}

\subsection{Product moment Correlation coefficient $r$}
\[
	r=\frac{S_{xy}}{\sqrt{S_{xx}S_{yy}}}
\]

The value of $r$ varies between -1 and 1\\

If $r=1$ , positive linear correlation\\

If $r=-1$, nagative linear correlation\\

If $r=0$, no linear correlation\\

limitation:

\subsection{Coding}
does not effect $r$
\section{Regression}
\subsection{Linear}
let $y=a+bx$ be a regression line \\
where
\[
	b=\frac{S_{xy}}{S_{xx}} \mbox{~and~} a=\bar{y}-b\bar{x}
\]
\subsection{Coding}
\subsection{Interpolation and Extrapolation}

\section{Discrete random variables}
\subsection{Probability distribution}

\begin{defi}[Mean / Expected value]
	\[
		E(X)=\sum xp(x)
	\]
\end{defi}
when we find $E(X^n)$:
\[
	E(X^n)=\sum x^np(x)
\]
\begin{defi}[Variable]
	\[
		Var(X)=E(X^2)-(E(X))^2
	\]
\end{defi}

The constant $a$ and $b$ affect on $E(X)$ and $Var(X)$
\[
	E(aX+b)=aE(x)+b
\]

\[
	Var(aX+b)=a^2Var(X)
\]
\begin{defi}[Uniform distribution]
	The distribution is uniform when all the probabilities is the same of all values.
\end{defi}


\section{The normal distribution}
$Z~ N(\mu,\sigma^2)$ represent the normal distribution.\\

\begin{center}
	\includegraphics[scale=0.5]{img_S/8_intro}
\end{center}

The random variable $X$ can be written as $X~ N(\mu,\sigma^2)$\\

you can transformed $X$ to $Z$ by this formula
\[
	z=\frac{X-\mu}{\sigma}
\]
\begin{eg}
    Example 8-9
\end{eg}

\section{Binomial distribution}

\section{Poisson distribution}

\section{Continuous random variables}

\section{Continuous uniform distribution}

\section{Normal approximation}

\section{Population and samples}

\section{Hypothesis testing}

\section{Combination of random variables}

\section{Sampling}

\section{Estimation , confidence intervals and tests}

\section{Goodness of fit and contingency tables}

\section{Regression and correlation}

\section{Quality of tests and estimators}

\section{One-sample procedures}

\section{Two-sample procedures}

\printindex




\end{document}}