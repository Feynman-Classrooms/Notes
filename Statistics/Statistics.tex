\documentclass[a4paper]{article}

\def\npart{III}
\def\nterm {Lent}
\def\nyear {2017-2018}
\def\nlecturer {Brian}
\def\ncourse {Statistics}

\input{header}

\begin{document}

\maketitle


\tableofcontents

\section{Representation and summary of data - location}
\subsection{Basic Concepts of Variable}
\begin{defi}[Quantitative variables and Qualitative variables]
	Quantitative variable associated with numerical observation. Qualitative variables associated with non-numerical observations.
\end{defi}

\begin{defi}[Continuous variable and discrete variable]
	Continuous variable can take ant value in given range. Discrete can take only specific values in a given range.
\end{defi}


\subsection{Grouped data}
\begin{defi}[Grouped data]
	The groups are more commonly known as classes.
	\begin{itemize}
		\item class boundaries.
		\item mid-point of a class.
		\item class width.
	\end{itemize}
\end{defi}

\begin{eg}
	Example 5-6
\end{eg}

\begin{defi}[Frequency and cumulative frequency]
	Number of anything; example is how many sheeps.
	It is sometimes helpful to add a column to the table showing the running total of the frequencies. This is called the cumulative frequency
\end{defi}


\begin{defi}[Ungrouped data]
	Show all data
\end{defi}

\subsection{Mean , mode and median}
\begin{defi}[Mode]
	The mode is the value that occurs most often
\end{defi}

\begin{defi}[Median]
	n/2 term or 1 term above
\end{defi}

\begin{defi}[Mean]
	\[
		\bar{x}=\frac{\sum_i^n x_i}{n}
	\]
\end{defi}

\subsection{Linear interpolation}
\begin{eg}
	Example 14-15
\end{eg}

\subsection{Coding}
\begin{eg}
	pick 1 example
\end{eg}


\section{Representation and summary of data - measures of dispersion}
\subsection{Range and interquartile range}
The list of formula:
\begin{itemize}
	\item Range = Upper value $-$ Lowest value
\end{itemize}
\begin{eg}
	example 3
\end{eg}

\subsection{Percentiles split the data into 100 parts}
\begin{eg}
	example 4
\end{eg}

\subsection{Range and Interquartile range}

\begin{eg}[Linear Interpolation]

\end{eg}

\subsection{Variance and standard deviation}
\begin{defi}[Variance]
	Let $f$ stand for the frequency, then $n=\sum f$ and
	\[
		\text{Variance}=\frac{\sum f(x-\bar{x})^2}{\sum f} \mbox{~or~} \frac{\sum fx^2}{\sum f}-(\frac{\sum fx^2}{\sum f})
	\]
\end{defi}

\subsection{Variance and standard deviation for grouped data}
\begin{defi}

\end{defi}

\begin{eg}
	example 7-8
\end{eg}

\subsection{Coding}

\begin{eg}
	example 9-11
\end{eg}

\section{Representation of data}

\subsection{Stem and Leaf diagrams}

\subsection{Outlier}
\begin{defi}

\end{defi}
\[
\]

\[
\]

\begin{eg}

\end{eg}

\subsection{Box plot}

\subsection{Histogram}

\begin{defi}[Frequency density]

\end{defi}

\subsection{Skewness (Shape)}

\subsection{What!?}
\section{Probability}
\subsection{Classical Probability}

\subsection{Venn diagram and their rules}
\begin{defi}[Complementary Probability]

\end{defi}

\subsection{Conditional Probabilites}
\subsubsection{Vann diagram}

\subsubsection{Tree diagram}

\subsection{Special Events of Probabilites}

\begin{defi}[Mutually exclusive]

\end{defi}

\begin{defi}[Independent events]

\end{defi}
\section{Correlation}
\subsection{Correlation}
\subsection{Bivariate data}
\begin{defi}[Co-Variance]

\end{defi}

\subsection{Product moment Correlation coefficient $r$}

\subsection{Coding}

\section{Regression}
\subsection{Linear}
\subsection{Coding}

\section{Discrete random variables}
\subsection{Probability distribution}

\begin{defi}[Variable]

\end{defi}
\begin{defi}[Expected value]

\end{defi}

\subsection{Coding}

\section{The normal distribution}

\section{Binomial distribution}

\section{Poisson distribution}

\section{Continuous random variables}

\section{Continuous uniform distribution}

\section{Normal approximation}

\section{Population and samples}

\section{Hypothesis testing}

\section{Combination of random variables}

\section{Sampling}

\section{Estimation , confidence intervals and tests}

\section{Goodness of fit and contingency tables}

\section{Regression and correlation}

\section{Quality of tests and estimators}

\section{One-sample procedures}

\section{Two-sample procedures}

\printindex




\end{document}}