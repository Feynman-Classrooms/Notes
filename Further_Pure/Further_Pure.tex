\documentclass[a4paper]{article}

\def\npart{III}
\def\nterm {Lent}
\def\nyear {2017-2018}
\def\nlecturer {Brian}
\def\ncourse {Further Pure Math}

\input{header}

\begin{document}
\maketitle




\tableofcontents

\section{Complex numbers}
\begin{eg}
Write $\sqrt{-36}$ in term of $i$
\begin{proof}
\begin{align*}
\sqrt{-36}=\sqrt{36*-1}=\sqrt{36}\times -1=6i
\end{align*}
\end{proof}
\end{eg}

\section{Numerical solutions of equations}

\section{Coordinate systems}

\section{Matrix algebra}

\section{Series}

\section{Proof by mathematical induction}
\subsection{Summation of series}
\begin{eg}
Prove by the method of mathematical induction , that, for $n\in\mathbb{Z}^+ , \sum_{r=1}^n(2r-1)=n^2$.
\begin{proof}
\begin{align*}
n=1;LHS&=\sum_{r=1}^1(2r-1)=2(1)-1=1\\
RHS&=1^2=1
\end{align*}
As $LHS=RHS$, the summation formula is true for $n=1$,\\

Assume hat the summation formula is true for $n=k$,
\begin{equation*}
i.e. \sum_{r=1}^k(2r-1)=k^2
\end{equation*}
With n=k+1 terms the summation formula becomes:
\begin{align*}
\sum_{r=1}^{k+1}(2r-1)&=1+2+3+\cdots+(2k-1)+(2(2k+1)-1)\\
&=k^2+(2(k+1)-1)\\
&=k^2+(2k+2-1)\\
&=k^2+2k+1\\
&=(k+1)^2
\end{align*}
Therefore , summation formula is true when $n=k+1$\\
If the summation formula is true for $n=k$ thenit is shown to be true for $n=k+1$. As the result is true for $n=1$, it is now also true for all $n\geq1$ and $n\in\mathbb{Z}^+$ by mathematical induction..
\end{proof}
\end{eg}

\begin{eg}
Prove by the method of mathematical induction, that , for $n\in\mathbb{Z}^+ , \sum_{r=1}^nr^2=\frac{1}{6}n(n+1)(2n+1)$
\begin{proof}
\begin{align*}
n=1; LHS&=\sum_{r=1}^1r^2=1^2=1\\
RHS&=\frac{1}{6}(1)(2)(3)=\frac{6}{6}=1
\end{align*}
As $LHS=RHS$, the summation formula is true for n=1\\

Assume that the summation formula is true for $n=k$
\begin{equation*}
i.e.\sum_{r=1}^kr^2=\frac{1}{6}k(k+1)(2k+1)
\end{equation*}
With $n=k+1$ terms the summation formula becomes:
\begin{align*}
\sum_{r=1}^{k+1}r^2&=1^2+2^2+3^2+\cdots+k^2+(k+1)^2\\
&=\frac{1}{6}k(k+1)(2k+1)+(k+1)^2\\
&=\frac{1}{6}(k+1)[k(2k+1)+6(k+1)]\\
&=\frac{1}{6}(k+1)[2k^2+k+6k+6]\\
&=\frac{1}{6}(k+1)[2k^2+7k+6]\\
&=\frac{1}{6}(k+1)(k+2)(2k+3)\\
&=\frac{1}{6}(k+1)(k+1+1)(2(k+1)+1)
\end{align*}
Therefore, summation formula is true when $n=k+1$\\
If the summation formula is true for $n=k$ thenit is shown to be true for $n=k+1$. As the result is true for $n=1$, it is now also true for all $n\geq1$ and $n\in\mathbb{Z}^+$ by mathematical induction..
\end{proof}
\end{eg}

\begin{eg}
Prove by the method of mathematical induction, that , for $n\in\mathbb{Z}^+ , \sum_{r=1}^nr^2=\frac{1}{6}n(n+1)(2n+1)$
\begin{proof}
\begin{align*}
n=1; LHS&=\sum_{r=1}^1r^2=1^2=1\\
RHS&=\frac{1}{6}(1)(2)(3)=\frac{6}{6}=1
\end{align*}
As $LHS=RHS$, the summation formula is true for n=1\\

Assume that the summation formula is true for $n=k$
\begin{equation*}
i.e.\sum_{r=1}^kr^2=\frac{1}{6}k(k+1)(2k+1)
\end{equation*}
With $n=k+1$ terms the summation formula becomes:
\begin{align*}
\sum_{r=1}^{k+1}r^2&=1^2+2^2+3^2+\cdots+k^2+(k+1)^2\\
&=\frac{1}{6}k(k+1)(2k+1)+(k+1)^2\\
&=\frac{1}{6}(k+1)[k(2k+1)+6(k+1)]\\
&=\frac{1}{6}(k+1)[2k^2+k+6k+6]\\
&=\frac{1}{6}(k+1)[2k^2+7k+6]\\
&=\frac{1}{6}(k+1)(k+2)(2k+3)\\
&=\frac{1}{6}(k+1)(k+1+1)(2(k+1)+1)
\end{align*}
Therefore, summation formula is true when $n=k+1$\\
If the summation formula is true for $n=k$ thenit is shown to be true for $n=k+1$. As the result is true for $n=1$, it is now also true for all $n\geq1$ and $n\in\mathbb{Z}^+$ by mathematical induction..
\end{proof}
\end{eg}

\subsection{Divisibility}

\subsection{General term of a recurrence relation}

\subsection{Matrix multiplication}

\section{Inequalities}

\section{Further Series}

\section{Further complex numbers}

\section{First order differential equations}

\section{Second order differential equations}

\section{Maclaurin and Taylor series}

\section{Polar coordinates}

\section{Hyperbolic functions}

\section{Further coordinate systems}

\section{Differentiation}

\section{Integration}

\section{Vectors - Cross product and vector equation of plane}

\section{Further matrix algebra}

\section{Extension topics - STEP}



\end{document}